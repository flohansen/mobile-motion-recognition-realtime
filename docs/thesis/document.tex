\documentclass{hsflensburg}
\title{Bewegungserkennung auf mobilen Geräten mit Verwendung von GANs für eine
automatische Datensatzgenerierung}
\subtitle{MASTER-THESIS}

\author{
  \name{Florian Hansen}\\
  \institution{Hochschule Flensburg}
}

\usepackage[ngerman]{babel}
\usepackage{csquotes}
\usepackage{biblatex}
\usepackage{amsmath}
\usepackage{amssymb}
\usepackage{mathtools}
\usepackage{tabularx}
\usepackage{amssymb}
\usepackage{multirow}
\usepackage{graphicx}
\usepackage{placeins}
\usepackage[locale=DE]{siunitx}
\usepackage[ngerman,linesnumbered,algoruled,boxed,lined]{algorithm2e}
\addbibresource{bibliography.bib}

\sisetup{
  group-digits=true,
  group-separator=\, ,
  group-minimum-digits=5,
  detect-all
}

\newtheorem{definition}{Definition}

\setcapindent{0pt}
\newcommand{\plh}{{\mkern-0.3mu\times\mkern-0.3mu}}

\newcommand{\resnetblocksimple}[3]{$\begin{bmatrix}3 \plh 3, #1 \\ 3 \plh 3, #2\end{bmatrix} \plh #3$}
\newcommand{\resnetbottleneck}[3]{$\begin{bmatrix}1 \plh 1, #1 \\ 3 \plh 3, #1 \\ 1 \plh 1, #2\end{bmatrix} \plh #3$}

\renewcommand\tabularxcolumn[1]{m{#1}}

\begin{document}
  \maketitle

  \chapter*{Eidesstattliche Erklärung}
{Ich versichere, dass ich die vorlegende Thesis ohne fremde Hilfe selbstständig verfasst und nur die angegebenen Quellen benutzt habe.\par}
\vspace{1.5cm}
{Flensburg, den 6. September 2021\par}
\vspace{1cm}
\begin{tabularx}{\linewidth}{p{6cm}X}\cline{1-1}
	Florian Hansen &
\end{tabularx}

  \tableofcontents
  \listoffigures
  \listoftables

  \chapter{Einleitung}
Kann eine Bewegungserkennung mithilfe von künstlichen neuronalen Netzen
effizient und in Echtzeit auf mobilen Geräten wie Smartphones ausgeführt
werden? Mit dieser Frage soll sich diese Arbeit beschäftigen. Speziell wird
sich dabei auf die Erkennung und Analyse von Bewegungen bezogen. Dabei müssen
bereits vorhandene Modelle umgewandelt werden, um auf mobilen Geräten eine
Echtzeiterkennung durchführen zu können. Künstliche Intelligenz hat bereits in
vielen verschiedenen Bereichen eine unterstützende Rolle eingenommen.
Dementsprechend ist das Feld in den letzten Jahren stetig gewachsen und hat an
Interesse gewonnen. Viele Anwendungen funktionieren nur deshalb, weil sie durch
Modelle des Machine-Learnings unterstützt werden. Vor allem in der
Computer-Vision findet diese Technologie Anwendung. Beispiele hierfür sind
Bildklassifizierer und Objekt-Detektoren, die entsprechend Bilder eine Klasse
zuordnen bzw. viele Objekte innerhalb eines Bildes erkennen. Neben der
Bildverarbeitung ist die Erkennung von menschlichen Posen bzw. von Bewegungen
mit künstlichen neuronalen Netzen (KNN) ein weiteres, aktuelles
Forschungsthema. Diese Art von Detektoren werden unter anderem dazu verwendet,
um Schlüs\-sel\-punkte des menschlichen Körpers zu identifizieren.

Während solche Modelle bereits im Desktopbereich mit weniger Einschränkungen
ausgeführt werden können, sind diese eher schwierig auf ressourcenarme Geräte
übertragbar. Oft müssen abgewandelte, verkürzte Varianten erstellt werden, um
die benötigte Rechenleistung so gering wie möglich zu halten -- die meisten
Smartphones haben zur Zeit leider nicht die gleichen Rechen- und
Speicherkapazitäten wie die meisten Desktopmaschinen, ganz zu schweigen von
diversen anderen Geräten des Internet-of-Things (IoT) wie Haushaltsgeräte und
Sensoren. Aus diesem Grund soll sich diese Arbeit insbesondere damit
beschäftigen, wie die Bewegungserkennung auf mobilen Geräten ausgeführt werden
kann. Zusätzlich wird untersucht, welche Anpassungen vorhandene
Machine-Learning-Modelle benötigen, um auf mobilen Geräte ausgeführt werden zu
können.

Kapitel \ref{chapter:basics} beschäftigt sich mit den Grundlagen der in dieser
Arbeit verwendeten Technologien. Dabei wird unter anderem darauf eingegangen,
wie Unterschiede zwischen Distributionen gemessen werden können, um damit den
Grundstein für spätere Loss-Funktionen zu schaffen. Diese werden dann vor allem
für das Trainieren von Generative-Adversarial-Networks (GANs) verwendet. Auch
werden in diesem Kapitel einige Grundbausteine zum Entwickeln von sehr tiefen
neuronalen Netzen besprochen. Anschließend wird die Funktionsweise von GANs und
entsprechenden Verlustfunktionen zum Trainieren dieser Netzwerke besprochen.
Zusätzlich werden Probleme der einzelnen Architekturen besprochen und Lösungen
vorgestellt.

Kapitel \ref{chapter:dataset} stellt Methoden vor, die zum Erstellen eines
Datensatzes verwendet werden. Dieser Datensatz wird anschließend verwendet, um
die Bewegungserkennung aus dem nächsten Kapitel zu implementieren und die
Modelle damit zu trainieren. Besonders wird in diesem Kapitel der Aufbau des
Datensatzes erläutert und inwiefern dieser mithilfe von GANs erweitert werden
kann. Insbesondere wird hier versucht, einen kompletten Datensatz mithilfe eines
GANs zu erzeugen, sodass dieses dynamisch anstelle eines echten Datensatzes
verwendet werden kann.

Kapitel \ref{chapter:motion-detection} führt die Bewegungserkennung ein und
vergleicht unter anderem verschiedene Ansätze zum Erkennen von menschlichen
Schlüsselpunkten. Dabei wird auf Single- und Multi-Pose-Detection eingegangen
und mit dessen Hilfe eine neue Netzwerkarchitektur definiert, die in der Lage
ist, eine Folge von menschlichen Posen zu klassifizieren und analysieren.

Kapitel \ref{mobile-app} beschäftigt sich mit der Entwicklung einer
Android-App, welche die ausgearbeiteten Ergebnisse aus den vorherigen Kapiteln
verwenden soll, um die Bewegungserkennung auf mobilen Geräten zu testen.
Hier soll die Forschungsfrage abschließend mithilfe beantwortet werden, indem
die entwickelte Bewegungserkennung auf verschiedene mobile Geräte getestet wird.

  \chapter{Grundlagen}\label{chapter:basics}
In diesem Kapitel sollen die Grundlagen, die in dieser Arbeit benötigt werden,
besprochen werden und bildet den Theorieteil der Thesis. Zuerst wird auf die
mathematischen Grundlagen eingegangen, die zum Definieren von Verlustfunktionen
für generative Machine-Learning-Modelle essentiell sind. Ziel dieses Teils ist
es die Frage zu beantworten, wie weit eine Wahrscheinlichkeitsverteilung von
einer anderen entfernt ist, um die Distanz anschließend zu minimieren. Das
Minimieren der Entfernung zwischen Verteilungen wird in darauf folgenden
Abschnitten verwendet, um Generative-Adversarial-Networks zu trainieren.

Neben den mathematischen Grundlagen werden einige grundlegende Bausteine für die
Entwicklung von hochqualitativen Merkmal-Extraktoren für die Objekterkennung auf
mobilen Endgeräten erläutert. Diese werden in späteren Kapiteln verwendet, um
neue neuronale Netzwerke zu entwickeln, die unter anderem menschliche Posen aus
Bildern sehr genau extrahieren und letztlich Bewegungen erkennen können. Hierfür
ist eine hohe Semantik bei einer hohen Qualität bzw. Auflösung unabdingbar, um
auch kleine Merkmale wie Hände erkennen zu können. Deshalb werden
Feature-Pyramid-Networks (FPNs) erläutert, die mithilfe eines sogenannten
Backbones hochauflösende Merkmale mit einer hohen Semantik extrahieren können,
während einfache Faltungsschichten, auch \textit{Convolutional-Layer} oder kurz
\textit{Conv} genannt, zwar ein hohes
Verständnis entwickeln, jedoch eine kleine Auflösung dieser Merkmale besitzen.

Neben den FPNs sollen auch die Architekturen von Residual-Neural-Networks
(ResNets) und Inverted-Residuals (MobileNetV2) erläutert werden. Diese dienen
meist als Backbones für Merkmal-Extraktoren und sind in Verbindung
mit FPNs essentieller Bestandteil der Posenerkennung.

\section{Notationen}
In dieser Arbeit werden verschiedene Notationen aus der Statistik und dem
Machine-Learning-Umfeld verwendet und sollen hier aufgrund der Lesbarkeit
aufgelistet werden.

\paragraph{Erwartungswert.}
Der Term $\mathbb{E}_{x \sim P}\left[f(x)\right]$ stellt den Erwartungswert
einer Verteilung $P$ dar und liest sich als \textit{erwarteter Wert von
$f(x)$ unter $x$ verteilt als $P$}.

\paragraph{Berechnung von Gradienten.}
Der Term $\nabla_w\left[f(x)\right]$ stellt die Berechnung der Gradienten von
den Parametern $w$ mithilfe der Loss-Funktion $f$ dar.

\paragraph{Euklidische Norm.}
Der Term $\| v \|$ stellt die euklidische Norm von $v$ dar. Sie ist definiert als die quadratische Wurzel der Summe aller Quadrate der Komponenten von $v$, also $\| v \| = \| v \|_2 = \sqrt{v_1^2 + v_2^2 + ... + v_n^2}$.

\section{Lipschitzstetigkeit}
\begin{definition}[K-Lipschitzstetigkeit]
Seien $(X, d_X)$ und $(Y, d_Y)$ metrische Räume. Eine Abbildung $f: X \to Y$
wird als K-lipschitzstetig bezeichnet, wenn
\[
    d_Y(f(x_1), f(x_2)) \leq K \cdot d_X(x_1, x_2)
\]
für alle $x_1, x_2 \in X$ gilt. $K$ wird hierbei als Lipschitzkonstante
bezeichnet und muss immer $K \geq 0$ erfüllen.
\end{definition}

\section{Kullback-Leibler-Divergenz}
Die Kullback-Leibler-Divergenz (KL-Divergenz) misst, wie sehr sich zwei
Verteilungen voneinander unterscheiden und hat seinen Ursprung in der
Informationstheorie. 
\begin{definition}[Kullback-Leibler-Divergenz \cite{arjovsky2017wasserstein}]
Seien $P$ und $Q$ zwei Wahrscheinlichkeitsfunktionen über den gleichen
Wahrscheinlichkeitsraum $X$. Dann ist der Abstand bzw. die Divergenz der
beiden Verteilungen definiert als
\[
    D_{KL}(P \lvert\lvert Q) = \sum_{x \in X} P(x) \log \frac{P(x)}{Q(x)}.
\]
\end{definition}
Dabei gibt $P \lvert\lvert Q$ eine Divergenz von der Ausgangsverteilung $P$
zur Zielverteilung $Q$ an. Das Messen der Divergenz zwischen zwei
Wahrscheinlichkeitsverteilungen findet insbesondere im Machine-Learning statt,
um künstliche neuronale Netze und ihre Gewichte zu trainieren. Deshalb kann
die KL-Divergenz auch als Loss-Funktion verwendet werden. Bemerkenswert ist
hierbei, dass die KL-Divergenz asymmetrisch ist, also $D_{KL}(P \lvert\lvert
Q) \neq D_{KL}(Q \lvert\lvert P)$. Die Distanz zwischen zwei Verteilungen
unterscheidet sich demnach je nach Ausgangsverteilung.

\section{Jensen-Shannon-Divergenz}
\begin{definition}[Jensen-Shannon-Divergenz \cite{arjovsky2017wasserstein}]
Seien $P$ und $Q$ zwei Wahr\-schein\-lichkeitsfunktionen über den gleichen
Wahrscheinlichkeitsraum $X$. Dann ist die Jensen-Shannon-Divergenz der
beiden Verteilungen definiert als
\[
    D_{JS}(P \lvert\lvert Q) = \frac{1}{2} D_{KL}(P \lvert\lvert M) + \frac{1}{2} D_{KL}(Q \lvert\lvert M) \quad\quad \text{mit} \;\; M = \frac{1}{2}(P + Q)
\]
\end{definition}
Die Jensen-Shannon-Divergenz kann als Erweiterung der
Kullback-Leibler-Divergenz angesehen werden. Im Gegensatz zur
Kullback-Leibler-Divergenz ist die Jensen-Shannon-Divergenz (JS-Divergenz)
symmetrisch. Das bedeutet, dass der Abstand zwischen zwei
Wahrscheinlichkeitsverteilungen gleich groß ist, egal von welchen der beiden
Distributionen aus betrachtet wird.

\section{Wasserstein-Abstand}
Eine weitere Methode zum Messen des Abstands zwischen zwei
Wahrscheinlichkeitsverteilungen ist die Berechnung des Wasserstein-Abstands.
Diese Methode wird besonders wichtig für die folgenden Abschnitte, in denen
generative neuronale Netze mithilfe des Wasserstein-Abstandes definiert und
umgesetzt werden.

\begin{definition}[Wasserstein-Abstand \cite{arjovsky2017wasserstein}]
Seien $P_r$ und $P_g$ zwei Wahrscheinlichkeitsverteilungen, dann ist der
Wasserstein-Abstand definiert als
\[
    W(P_r, P_g) = \inf_{\gamma \in \Pi(P_r, P_g)} \mathbb{E}_{(x, y) \sim \gamma} \left[\|x - y\|\right],
\]
wobei $\Pi(P_r, P_g)$ die Menge aller gemeinsamen Verteilungen $\gamma(x,
y)$ darstellt, dessen Grenzen $P_r$ und $P_g$ sind.
\end{definition}

Der Term $\gamma(x, y)$ stellt dabei die \textit{Masse} dar, die von $x$ nach
$y$ transportiert wird, um schließlich die Verteilung $P_r$ in die Verteilung
$P_g$ umzuformen. Aus diesem Grund ist der Wasserstein-Abstand auch als
\textit{Earth-Mover-Abstand} (EM-Abstand) bekannt.

\FloatBarrier
\section{Backbones}
Backbones sind Netzwerke, die vor allem Merkmale (Features) aus der Eingabe, wie
z.B. aus einem Bild, extrahieren, ohne dass eine Abschätzung durchgeführt wird.
Die extrahierten Merkmale werden anschließend verwendet, um beispielsweise eine
Abschätzung mithilfe eines Regressors auszuführen \cite{amjoud2020backbones}.
Man kann demnach verschiedene Backbones mit unterschiedlichen Regressoren
verbinden, ganz nach dem Baukastenprinzip. In dieser Arbeit werden vor allem
ResNet- und MobileNet-Backbones verwendet, da einige Konfigurationen von ihnen
vielversprechend für die Ausführung auf mobilen Geräten sind. Abbildung
\ref{fig:backbone-regressor-framework} soll verdeutlichen, wie
Machine-Learning-Modelle mithilfe eines solchen modularen Frameworks
zusammengestellt werden können.

\begin{figure}
  \centering
  \includegraphics[width=0.9\textwidth]{images/backbone-regressor-framework.pdf}
  \caption{Schematische Darstellung einer Machine-Learning-Architektur, die aus
  einem Backbone für die Merkmal-Extraktion und einem Prediction-Head für die
Klassifizierung bzw. Objektdetektion besteht. Optional können
Feature-Extraktoren wie Feature-Pyramid-Networks o.ä. an das Backbone gehängt
werden, um z.B. die Qualität der Merkmale zu verbessern.}
  \label{fig:backbone-regressor-framework}
\end{figure}

\FloatBarrier

\section{Residual-Neural-Networks}\label{section:residual-blocks}
Mit \cite{he2015deep} wurde ein neues Framework zum Trainieren von tiefen
neuronalen Netzen vorgestellt. Ein großes Problem mit sehr tiefen neuronalen
Netzen ist, dass diese zu einem größeren Fehler im Training führen. Dies führt
ebenfalls zu einem erhöhten Test-Fehler. Dies steht im Konflikt mit der
eigentlichen Vermutung, dass tiefere Netzwerke intelligenter sein müssten, als
weniger tiefe. Diese Fehler werden überraschenderweise immer größer, je tiefer
das Netzwerk ist, was als \textit{Degradation} bezeichnet wird.
Dies ist auch bei der Genauigkeit solcher Netze beobachtbar. Ist die
Genauigkeit gesättigt und erhöht man nun die Tiefe des Netzes, so degradiert
die Genauigkeit. Laut \cite{he2015deep} liegt dies nicht am Overfitting
(Überanpassung). Overfitting beschreibt die Überspezialisierung eines
Machine-Learning-Modells, welches sich zu sehr an den dahinterliegenden
Datensatz angepasst hat. Ein solches Modell kann nicht mehr zuverlässig mit
Daten außerhalb des zum Trainieren verwendeten Datensatzes betrieben werden.
Anders ausgedrückt besitzt das Netzwerk eine sehr hohe Genauigkeit beim
Arbeiten mit dem Trainingsdatensatz, aber eine signifikant niedrigere
Genauigkeit beim Arbeiten mit dem Testdatensatz.

Abhilfe für Degradation sollen die Residual-Networks (ResNets) mithilfe von
\textit{Building-Blocks} schaffen. Anstatt die Abbildung
$F(x)$ bei normalen gestapelten Schichten zu optimieren, wird bei ResNet das
Residuum $F(x) + x$ modelliert. Dies kann mithilfe von Shortcut-Connections
(Abkürzungsverbindungen) realisiert werden (siehe Abbildung
\ref{fig:resnet-building-block}).

\begin{figure}
    \centering
    \includegraphics[width=0.8\textwidth]{images/resnet_building_block.pdf}
    \caption{Schemata für Building-Blocks eines ResNets \cite{he2015deep}. Die
    Shortcut-Connection ist die Identität von dem Eingabeparameter $x$. Links ist ein einfacher Residual-Block dargestellt, während rechts ein tieferer, soganannter Bottleneck-Block dargestellt wird.}
    \label{fig:resnet-building-block}
\end{figure}

Abhängig von den Anforderungen können nun entsprechend tiefe Netze eingesetzt
werden, indem die Building-Blocks verkettet eingesetzt werden. In
\cite{he2015deep} werden außerdem Architekturen vorgestellt, die
erfolgreich auf den ImageNet-Datensatz \cite{deng2009imagenet} evaluiert wurden.
Diese unterscheiden sich hauptsächlich in der Anzahl der verwendeten Schichten
und sind in Tabelle \ref{table:resnets} zu sehen. Jeder dieser Residual-Blöcke,
egal ob einfacher Building- oder Bottleneck-Block, kann einen Stride besitzen. Bottleneck-Blöcke sind tiefere Residual-Blöcke mit drei Faltungsschichten, deren Filter einen Flaschenhals darstellen (vgl. Abbildung \ref{fig:resnet-building-block}).
Dabei ist wichtig, dass der entsprechende Stride nur auf die erste
Faltungsschicht des gesamten Blocks angewandt wird. Die restlichen Schichten
besitzen demnach immer einen Stride von 1. Die Ergebnisse nach dem Trainieren
dieser Netze haben gezeigt, dass das Degradation-Problem gelöst werden konnte,
sodass tiefere Netze mithilfe von Residual-Blocks auch einen geringeren Fehler
erzeugen, also genauer arbeiten als weniger tiefe Netze.

\begin{table}
    \scriptsize
    \begin{tabularx}{\textwidth}{X|X|c|c|c|c|c}
        \hline
        layer name & output size & 18-layer & 34-layer & 50-layer & 101-layer & 152-layer \\ \hline
        conv1 & $112 \times 112$ & \multicolumn{5}{c}{$7 \times 7$, 64, stride 2} \\ \hline
        \multirow{2}{*}{conv2\_x} & \multirow{2}{*}{$56 \times 56$} & \multicolumn{5}{c}{$3 \times 3$, max pool, stride 2} \\ \cline{3-7}
        & & \resnetblocksimple{64}{64}{2} & \resnetblocksimple{64}{64}{3} & \resnetbottleneck{64}{256}{3} & \resnetbottleneck{64}{256}{3} & \resnetbottleneck{64}{256}{3} \\ \hline
        conv3\_x & $28 \times 28$ & \resnetblocksimple{128}{128}{2} & \resnetblocksimple{128}{128}{4} & \resnetbottleneck{128}{512}{4} & \resnetbottleneck{128}{512}{4} & \resnetbottleneck{128}{512}{8} \\ \hline
        conv4\_x & $14 \times 14$ & \resnetblocksimple{256}{256}{2} & \resnetblocksimple{256}{256}{6} & \resnetbottleneck{256}{1024}{6} & \resnetbottleneck{256}{1024}{23} & \resnetbottleneck{256}{1024}{36} \\ \hline
        conv5\_x & $7 \times 7$ & \resnetblocksimple{512}{512}{2} & \resnetblocksimple{512}{512}{3} & \resnetbottleneck{512}{2048}{3} & \resnetbottleneck{512}{2048}{3} & \resnetbottleneck{512}{2048}{3} \\ \hline
        & $1 \times 1$ & \multicolumn{5}{c}{average pool, 1000-d fc, softmax} \\ \hline
        \multicolumn{2}{c|}{FLOPs} & \num{1.8e9} & \num{3.6e9} & \num{3.8e9} & \num{7.6e9} & \num{11.3e9} \\ \hline
    \end{tabularx}
    \caption{Architekturen für ImageNet \cite{he2015deep}. Es werden die gestapelten Schichten bzw. Building-Blocks für ResNet-18, -34, -50, -101 und -152 aufgelistet. Die Building-Blocks werden hier als Matrix dargestellt und geben pro Zeile die Kernelgröße und Anzahl der Kanäle der Faltungsschichten an. Das Multiplikationszeichen hinter den Matrizen gibt die Anzahl der verketteten Building-Blocks an.}
    \label{table:resnets}
\end{table}

\section{MobileNetV2}
Dieser Abschnitt soll nun eine Architektur vorstellen, die geeignet ist, um
Modelle des maschinellen Lernens auf mobilen Geräten auszuführen. Diese soll
als Grundlage zum Verständnis der MoveNet-Architektur aus \cite{movenet}
dienen, die später für die Bewegungserkennung wichtig wird. Die Rede ist
hierbei von MobileNetV2 \cite{sandler2019mobilenetv2} und den damit
eingeführten Inverted-Residual-Blocks. Diese besitzen einen verringerten
Speicherverbrauch bei einer Inferenz, was für mobile Anwendungen sehr wichtig
ist und sind wie folgt aufgebaut. Die Eingabe $x$ in einem solchen Block wird
zuerst in eine Faltungsschicht mit einem $1 \times 1$ Kernel und einer
ReLU6-Aktivierung gegeben. Danach wird eine Depthwise-Separable-Convolution
\cite{howard2017mobilenets} durchgeführt, die eine Kernelgröße von $3 \times 3$
verwendet. Auch hier wird eine ReLU6-Aktivierung vorgenommen. Die nächste
Schicht ist wieder eine $1 \times 1$ Faltungsschicht mit einer linearen
Aktivierungsfunktion. Zum Schluss wird die Ausgabe der letzten Faltungsschicht
mit der Identität der Eingabe $x$ addiert und bildet ein Residuum. Der gesamte
Ablauf ist in Abbildung \ref{fig:inverted-residual} zu sehen und ähnelt sehr
stark den Residual-Blocks aus Abschnitt \ref{section:residual-blocks}. Neben
den üblichen Parametern wurden ein paar wenige hinzugefügt. So beschreibt der
Expansionsfaktor $t$ einen Skalar, der die Anzahl der internen Filter des
Blocks skaliert. Dieser hat keinen Einfluss auf die Eingabe- oder Ausgabegröße.
Das beudetet, dass bei einer Eingabegröße von $h \times w \times c$ die erste
und zweite Faltungsschicht $c \cdot t$ Ausgabefilter besitzen. Auf die dritte
und letzte Schicht eines Inverted-Residual-Blocks hat der Expansionsfaktor
keinen direkten Einfluss. Hier wird lediglich von $c \cdot t$ auf $c'$ Filter
projeziert, indem eine Faltungsschicht mit $c'$ Filtern verwendet wird. Bei dem Stride eines kompletten Blocks verhält es sich ähnlich.
Dieser wird lediglich auf die zweite also
Depthwise-Separable-Convolution-Schicht angewandt. Die restlichen Schichten
besitzen einen Stride von 1.

Das MobileNetV2 ist relativ einfach aufgebaut. Die erste Schicht
ist eine Faltungsschicht gefolgt von 17 Inverted-Residual-Blocks
und einer weiteren Faltungsschicht. Der gesamte Aufbau soll
durch Tabelle \ref{table:mobilenetv2} dargestellt werden. Wie schon eingangs
erwähnt, wird dieses Modell später für die Bewegungs- bzw. Posenerkennung
verwendet. Genauer gesagt wird das MobileNetV2 als Backbone für einen
Merkmal-Extraktor verwendet. Dies soll jedoch in dem nächsten Abschnitt näher
erläutert werden.

\begin{table}
    \begin{tabularx}{\textwidth}{l|l|l|l|l|l}
        \hline
        Input & Layer & t & c & n & s \\
        \hline
        $224 \times 224 \times 3$ & Conv2D, $3 \times 3$ & - & 32 & 1 & 2 \\
        $112 \times 112 \times 32$ & Bottleneck & 1 & 16 & 1 & 1 \\
        $112 \times 112 \times 16$ & Bottleneck & 6 & 24 & 2 & 2 \\
        $56 \times 56 \times 24$ & Bottleneck & 6 & 32 & 3 & 2 \\
        $28 \times 28 \times 32$ & Bottleneck & 6 & 64 & 4 & 2 \\
        $14 \times 14 \times 64$ & Bottleneck & 6 & 96 & 3 & 1 \\
        $14 \times 14 \times 96$ & Bottleneck & 6 & 160 & 3 & 2 \\
        $7 \times 7 \times 160$ & Bottleneck & 6 & 320 & 1 & 1 \\
        $7 \times 7 \times 320$ & Conv2D, $1 \times 1$ & - & 1280 & 1 & 1 \\
        \hline
    \end{tabularx}
    \caption{Aufbau von MobileNetV2 nach \cite{sandler2019mobilenetv2}. Der
    Bottleneck-Block entspricht einem Inverted-Residual-Block (siehe Abbildung
    \ref{fig:inverted-residual}). Der Parameter $t$ beschreibt den
    Skalierungsfaktor, $c$ die Anzahl der Filter, $n$ die Anzahl der aneinander
    geketteten Blöcke und $s$ den Stride eines Block-Verbunds mit $n$ Blöcken.}
    \label{table:mobilenetv2}
\end{table}

\begin{figure}
    \centering
    \includegraphics[width=0.3\textwidth]{images/inverted_residual.pdf}
    \caption{Darstellung eines Inverted-Residual-Blocks. Der \textit{Dwise}-Block stellt eine Depthwise-Separable-Convolution aus \cite{howard2017mobilenets} dar. Der gesamte Block wird auch als Bottleneck-Block in der MobileNetV2-Architektur bezeichnet. Nicht zu verwechseln mit dem Bottleneck-Block aus dem ResNet-Kontext.}
    \label{fig:inverted-residual}
\end{figure}

\section{Feature-Pyramid-Networks}
Bei der Objekterkennung ist beim Erlernen der Merkmale eines Bildes die
Auflösung dieser Merkmale sehr gering. Feature-Pyramid-Networks (FPN) haben die
Aufgabe, eine hohe Semantik bei einer hohen Auflösung zu generieren. Häufig sind
diese Netzwerke nur Teil eines Backbones bzw. werden dahinter geschaltet. Als
Backbone-Modell kann zum Beispiel AlexNet, MobileNet und ResNet dienen. In
Kombination mit einem FPN werden diese Netze damit zu einem Merkmal-Extractor
umgewandelt, der eine hohe Semantik bei einer hohen Auflösung der Merkmale
erlernen kann. Der Weg von der Eingabe über das Backbone-Modell wird auch als
\textit{Bottom-Up-Pathway} bezeichnet, während der Weg vom Backbone über die
Feature-Pyramid als \textit{Top-Down-Pathway} bezeichnet wird
\cite{lin2017feature}.

Anfänglich wurden FPNs in Verbindung mit ResNet-Backbones eingeführt (siehe
Abbildung \ref{fig:resnet-fpn}). Die Eingabe ist ein $224 \times 224 \times 3$
Bild und wird mithilfe der ersten ResNet-Schicht (C1) in $112 \times 112 \times
64$ Filter mithilfe eines Convolutional-Layers mit einem Stride von 2 und einem
$7 \times 7$ Kernel umgeformt. Anschließend wird die Ausgabe in ein
Max-Pooling-Block gegeben, welcher die Eingabe in $56 \times 56 \times 128$
Filter umwandelt. Die nächsten Blöcke sind für die Einbettung von FPNs am
wichtigsten, denn sie bilden eine Schnittstelle, die verwendet wird, um ein FPN anzubinden. Der folgende Block (C2) besteht aus mehreren Residual-Blocks,
welche zusammen einen Stride von 1 ergeben und 256 Filter erzeugen. Diese Blöcke
werden zusammengefasst auch \textit{Bottlenecks} genannt. Darauf folgen drei
weitere Bottlenecks (C3, C4, C5) jeweils mit einem Stride von 2. Nach C5 ensteht
somit eine Ausgabe von $7 \times 7 \times 2048$. Soweit zum Aufbau des
Bottom-Up-Pathways. Der Top-Down-Pathway wird mithilfe von Verbindungsschichten
(Laterals) mit dem Backbone (C2, C3, C4, C5) verbunden. Diese haben die Aufgabe, die Anzahl der
Filter aus dem Bottom-Up-Pathway anzugleichen, sodass die Ausgaben aus dem Bottom-Up-Pathway mit denen des Top-Down-Pathways addiert werden können. Hierfür werden
Convolutional-Layer mit einem $1 \times 1$ Kernel und 256 Filtern verwendet,
sodass lediglich die Anzahl der Filter transformiert werden. Im Konkreten
bedeutet dies, dass die Ausgabe von C5 von $7 \times 7 \times 2048$ auf die Form
$7 \times 7 \times 256$ gebracht wird.  Die Größe dieser Ausgabe wird nun
mithilfe von Upsampling (M5) verdoppelt und mit der Ausgabe der
Lateralverbindung von C4 addiert (M4). Dies wird mit den übrigen
Lateralverbindungen und Bottleneck-Blöcken verkettet wiederholt, sodass das FPN
schließlich vier Ausgaben mit den Größen $56 \times 56 \times 256$, $28 \times
28 \times 256$, $14 \times 14 \times 256$ und $7 \times 7 \times 256$ (M5, M4,
M3, M2) ausgibt. Betrachtet man nun die letzte Ausgabe M5 relativ zum
Eingabeformat, so besitzt diese Architektur einen Stride von $\frac{224}{56} = 4$.

Kurz zusammengefasst wird eine große Eingabe mithilfe des Bottom-Up-Pathways
bzw. Backbones zu vielen kleinen Filtern umgeformt, um die Merkmale des Bildes
zu verstehen. Der Top-Down-Pathway versucht hingegen diese relativ kleinen Filter hochzuskalieren. Dadurch sollen die hohe Semantik aus dem Bottom-Up-Pathway beibehalten werden und die Filter eine höhere Auflösung erhalten. Das Problem beim
Vergrößern der Filter ist, dass dabei ein Alias-Effekt auftritt, das Bild also
verschwommen wirkt. Hierfür dienen Smoothing-Layer, welche wiederum nichts
weiter als Convolutional-Layer mit einem $3 \times 3$ Kernel und einem Stride
von 1 sind. Entsprechend werden die Ausgaben M5 - M2 verschärft und ergeben
Merkmale mit einer hohen Auflösung und einer hohen Semantik, die nun wahlweise
für die Objekterkennung verwendet werden können.

\begin{figure}
    \includegraphics[width=\textwidth]{images/ResNet_FPN.pdf}
    \caption{Architektur eines Merkmal-Extractors als Feature-Pyramid-Network
    mit ResNet als Backbone.}
    \label{fig:resnet-fpn}
\end{figure}

Ähnlich verhält sich der Vorgang beim Verwenden eines MobileNetV2-Backbones, wie
in Abbildung \ref{fig:mobilenetv2-fpn} dargestellt. Da das MobileNetV2 mehr Blöcke
als das ResNet besitzt, die Anzahl der Pyramidenstufen aber gleich bleiben soll, werden
einige Blöcke übersprungen und sind nicht direkt Teil des Top-Down-Pathways. Im Prinzip
werden lediglich die Blöcke mit Lateralverbindungen mit dem Top-Down-Pathway verknüpft,
welche die letzte Stufe vor der Halbierung der Filtergröße darstellen. Also immer dann,
wenn der Stride 2 ist. Ausnahme ist C5. Hier wird einfach die letzte Schicht des Netzwerks
verwendet. Der Einbau eines FPN erfolgt ansonsten wie vorher erläutert. Die Kombination mit
MobileNetV2 und Feature-Pyramid-Network ist in dieser Arbeit deshalb wichtig, da dieser
Merkmal-Extraktor mit für die Posenerkennung zuständig ist. Besonders, weil die
Merkmale
eine hohe Semantik und eine hohe Auflösung besitzen, können besonders kleine Objekte innerhalb
eines Bildes erkannt und benötigte Informationen extrahiert werden. In Kapitel \ref{chapter:motion-detection}
wird die Posenerkennung genauer erläutert.

\begin{figure}
    \includegraphics[width=\textwidth]{images/MobileNetV2_FPN.pdf}
    \caption{Architektur eines Merkmal-Extractors als Feature-Pyramid-Network
    mit MobileNetV2 als Backbone.}
    \label{fig:mobilenetv2-fpn}
\end{figure}

  \section{Generative-Adversarial-Networks}\label{chapter:gans}
In Machine-Learning existieren viele verschiedene Modelle, die vorhandene
Datensätze analysieren und anhand der Daten lernen, Strukturen in den
Datensätzen zu erkennen.  Besitzt man beispielsweise einen Datensatz
bestehend aus Fotoaufnahmen von Tieren, so kann ein Klassifizierer trainiert
werden, um einem Bild eine Tierklasse zuzuweisen. Aus diesem Grund fässt man
diese Modelle unter dem Begriff \textit{Bildklassifizierung} \cite{lorente2021image} zusammen.

Auch interessant ist das Erkennen von vielen Objekten innerhalb eines
Bildes, anstatt das gesamte Bild nur einer einzigen Klasse zuzuweisen. In der
\textit{Objekterkennung} entwickelt man Modelle, welche mehr als nur eine
Klasse erkennen können. Sie liefern zusätzlich zu den erkannten Klassen ihre
Position und Größe innerhalb des Bildes. Diese Modelle treffen also keine
Aussage über das Gesamtbild, sondern treffen Aussagen über einzelne Objekte
innerhalb des Bildes \cite{zaidi2021survey}.

Neben Modellen, die zu einem bestimmten Sachverhalt eine Aussage treffen
können, existieren auch Modelle, welche in der Lage sind, neue Sachverhalte zu
erzeugen. Diese fallen unter dem Begriff \textit{Generative Adversarial
Networks} (GANs) und bilden das Hauptthema dieses Abschnitts. Das interessante
an diesen generativen Modellen ist, dass sie nicht nur die Strukturen eines
Datensatzes lernen, sondern darüber hinaus neue Elemente der
Ausgangsdistribution erzeugen können. Trainiert man also ein generatives
Modell auf einen Datensatz, welcher Bilder von verschiedenen Tieren enthält,
können neue Bilder der gleichen Art erzeugt werden.

Aber nicht nur zum Erzeugen von Bildern kann diese Art von Modellen verwendet
werden. Auch bei Aufgaben, bei denen eine Voraussagung getroffen werden soll,
werden generative Modelle eingesetzt. Beispielsweise wurde in
\cite{barsoum2017hpgan} gezeigt, wie zu bereits getätigten menschlichen
Bewegungen unterschiedliche, darauf folgende Bewegungssequenzen aussehen
können. Hier hat man also versucht, eine Vorhersage zur Entwicklung von
menschlichen Bewegung zu tätigen.

Die Funktionsweise von GANs ist im Prinzip ziemlich simpel. Während beim
klassischen überwachten Lernen (Supervised-Learning) in der Regel nur ein
Modell beim Training involviert ist, verhält sich das bei generativen Modellen
etwas anders. Zum Einen wird ein Generator definiert, welcher, wie sein Name
andeutet, Ausgaben selbst erzeugt. Zum Anderen wird ein Diskriminator in das
Training eingebaut, welcher zwischen künstlich erzeugten und realen Daten
unterscheidet. Diese beiden Modelle werden dann gleichermaßen trainiert.
Während der Generator versucht, immer bessere Fälschungen zu erzeugen, versucht
der Diskriminator immer besser zwischen Fälschung und Realität zu
unterscheiden. Die Ausgabe des Diskriminators ist dementsprechend entweder 0
für Fälschung und 1 für Realität. Mit anderen Worten, die beiden Komponenten
spielen ein Spiel, in welchem die eine Partei versucht, die andere zu täuschen
und wird mithilfe des folgenden Ausdrucks definiert \cite{goodfellow2014generative}.
\[
\min_G \max_D V(G, D) = \mathbb{E}_{x \sim p_{data}(x)}\left[ \log D(x) \right] + \mathbb{E}_{z \sim p_z(z)}\left[ \log (1 - D(G(z))) \right]
\]

Der Eingabeparameter $z \in \mathbb{R}^n$ stellt einen $n$-dimensionalen Vektor
dar, welcher auch als latenter Vektor bezeichnet wird. Dementsprechend wird der
Vektorraum auch als latenter Raum bezeichnet. Ein solcher Vektor repräsentiert
verschiedenste Objekte mithilfe seiner $n$ Komponenten. In Bezug zu GANs wurde
festgestellt, dass ein GAN nicht nur lernt, ähnliche Daten zu einem bestehenden
Datensatz zu erzeugen, sondern auch den Zusammenhang zwischen latenten
Komponenten und einer Ausgabe zu verstehen. Das schöne daran ist, dass diese
latenten Vektoren mithilfe der Gesetze aus der linearen Algebra analysiert und
entsprechende Operationen mit ihnen ausgeführt werden können. Hierzu kann
folgendes, sehr einfaches Experiment durchgeführt werden. Man wählt zwei
latente Vektoren $a, b \leftarrow \mathbb{R}^n$, wobei die Komponenten dieser
Vektoren gleich sind, also $a_i = b_i, \; 0 \leq i < n$. Nun wählt man einen
zufälligen Komponentenindex $j \in \mathbb{N}$ mit $j \in \left[0, n\right[$
und wählt einen zufälligen Wert für die Komponenten beider Vektoren $a_j
\leftarrow \mathbb{R},\; b_j \leftarrow \mathbb{R}$, wobei $a_j \neq b_j$.
Betrachtet man nun die Ausgaben des Generators mit $a$ und $b$ als Eingabe,
dann unterscheiden sich diese um die Eigenschaft, die durch die Komponente an
der Stelle $j$ beeinflusst wird. Das gleiche Prinzip kann auch rückwärts
durchgeführt werden. Erzeugt der Generator zum Beispiel ein Bild, worauf ein
Gesicht mit Brille zu sehen ist und ein weiteres Bild mit Gesicht ohne
Brille, dann kann man die beiden Eingabevektoren voneinander subtrahieren und
damit die Komponenten herausfinden, die für die entsprechenden Eigenschaften
zuständig sind (in diesem Fall, ob das Gesicht eine Brille trägt oder nicht).

Im Verlauf des Trainings entwickelt sich damit ein Generator, welcher im
Idealfall so gute Fälschungen erzeugt, sodass sich diese nicht mehr von Daten
der Ausgangsdistribution unterscheiden lassen. Der Diskriminator kann hier
bestenfalls nur raten, also eine Genauigkeit von höchstens 50\%
erreichen. Ist dies nicht der Fall, d.h. der Diskriminator kann Fälschungen
mit einer höheren Wahrscheinlichkeit von realen Daten unterscheiden, so
entsteht ein Ungleichgewicht. Aus diesem Grund sollten die Lernparameter
sorgfältig ausgewählt und untersucht werden, damit ein stabil laufendes GAN
trainiert wird.

\subsection{Das Mode-Collapse-Problem}
Ein großes Problem beim Trainieren von generativen neuronalen Netzen ist, dass
sich der Generator sehr häufig auf bestimmte Merkmale der Ausgangsdistribution
des Datensatzes fixiert. Das Ergebnis sind signifikant erhöht wiederkehrende
Ergebnisse, die sich kaum bis gar nicht von anderen Ausgaben unterscheiden.
Man erwartet jedoch, dass das jeweilige GAN eine vielseitige Variation aus
allen Elementen des Datensatzes erzeugt. Mit anderen Worten, bei einer
zufälligen Eingabe in das Netz, soll immer eine unterschiedliche Ausgabe
erzeugt werden. Bei einem Mode-Collapse ist dies nicht der Fall. Es kann
beispielsweise passieren, dass wenn das Netz auf das Erzeugen von neuen
Gesichtern trainiert wird, dass dieses ausschließlich weibliche Gesichter
erzeugt, weil das Netz herausgefunden hat, dass es einfacher ist, weibliche
Gesichtszüge zu generieren, als männliche \cite{richardson2018gans}. Dies
lässt sich damit erklären, dass der Generator beim Trainingsvorgang mehr
Erfolg beim Generieren von weiblichen Gesichtern hatte und der Diskriminator
es schwerer hatte, Fälschung von Realität zu unterscheiden. Um das Problem zu
beseitigen wurden einige Änderungen an dem Standardmodell des GAN von
\cite{goodfellow2014generative} vorgenommen. Wie dieses Problem gelöst wird,
soll in den nächsten Abschnitten erläutert werden.

\subsection{Deep-Convolution-GAN}
Das \textit{Deep Convolution GAN} (DCGAN) ist ein Versuch,
\textit{Convolutional Neural Networks} (CNNs) mit GANs zu verknüpfen. Nach
vielen Fehlschlägen in der Entwicklung von GANs mit CNNs ist die Version von
\cite{radford2016unsupervised} stabil und auf viele unterschiedliche
Datensätze anwendbar. Dafür wurden viele verschiedene Kombinationen von
Schichten untersucht und es wurde dabei eine Architektur ausgearbeitet, die
in ein stabiles Training über verschiedenste Datensätze resultierte.
Zusätzlich können mithilfe dieser Architektur höhere Auflösungen und tiefere
Netze erreicht werden.

Zusätzlich zur eigentlichen Architektur von DCGAN werden moderne Techniken
verwendet, um CNN-Architekturen zu vereinfachen.  Damit der Generator über
mehrere Schichten hinweg die räumliche Darstellung von Objekten lernen kann,
werden Convolutional-Layer verwendet. Anstatt, dass sogenannte
Max-Pooling-Layer zum Einsatz kommen, können nach
\cite{springenberg2015striving} einfach Convolutional-Layer mit erhöhtem
Stride verwendet werden, ohne dass die Genauigkeit sinkt. In Bezug zu DCGANs
von \cite{radford2016unsupervised} werden solche Schichten verwendet, um dem
Generator das Erlernen vom räumlichen Upsampling zu ermöglichen. Auch der
Diskriminator wird mit solchen CNN-Layer ausgestattet, um räumliches
Downsampling zu erlernen.

Neben dem Auslassen von Max-Pooling-Layer folgt DCGAN auch dem Trend,
Fully-Connected-Layer vor jedem Convolutional-Feature zu vermeiden. Dabei
wurde festgestellt, dass die Verknüpfung von Fully-Connected-Layer und der
Eingabe des Generators bzw. mit der Ausgabe des Diskriminators am besten
funktionieren. Die erste Schicht des Generators ist also ein
Fully-Connected-Layer (1-dimensional), jedoch wird die Ausgabe der Schicht in
einen 4-dimensionalen Tensor umgewandelt. Im Falle des Diskriminators wird die
Ausgabe des letzen Convolutional-Layers (4-dimensional) abgeflacht und in eine
1-dimensionale Schicht mit einer Sigmoid-Aktivierungsfuntion gefüttert
\cite{radford2016unsupervised}.

Um Mode-Collapse zu vermeiden, verwendet \cite{radford2016unsupervised}
Batch-Normalization-Layer. Dadurch wird das Training stabilisiert und Probleme
wie \textit{Internal-Covariate-Shifting} angegangen \cite{pmlr-v37-ioffe15}.
Vor allem wird dadurch aber auch verhindert, dass der Generator immer die
gleichen Ausgaben erzeugt. Das Anwenden der Batch-Normalisierung in allen
Schichten des Netzwerks führt jedoch zur Stichprobenoszillation und
Instabilität des Modells. Aus diesem Grund wird auf Batch-Normalization in der
Ausgabesschicht des Generators und in der Eingabeschicht des Diskriminators
verzichtet.

Als letzte Beobachtung stellt \cite{radford2016unsupervised} fest, dass das
Hinzufügen von ReLU-Ak\-ti\-vier\-ungs\-funk\-tio\-nen in allen Schichten des
Generators zu schnellerem Lernen und Abdeckung der Farbräume der
Trainingsdistribution führt. In der Ausgabeschicht wird jedoch anstatt von
ReLU-Aktivierung eine Tanh-Aktivierung verwendet. Innerhalb des Diskriminators
werden schließlich Leaky-ReLU-Aktivierungen angewandt.

\begin{figure}
\includegraphics[width=\textwidth]{images/dcgan-architecture}
\caption{DCGAN-Architektur des Generators von
\cite{radford2016unsupervised}. Als Eingabe dient ein 100-dimensionaler
Vektor, dessen Elemente zufällig gewählt werden. Dieser wird dann in den
ersten Schichten umgeformt und durch vier Convolutional-Layer auf die Form
3$\times$64$\times$64 gebracht. Die Strides geben dabei den
Vergrößerungsfaktor pro Convolution-Schicht an, während die Anzahl der
Filter den Farbkanälen entsprechen.}
\end{figure}

\subsection{Wasserstein-GAN}
Anders als andere GAN-Varianten verwendet das Wasserstein-GAN (WGAN) die
Was\-ser\-stein-Distanz anstelle der JS- oder KL-Divergenz, um die Gewichte
von generativen neuronalen Netzen zu optimieren. Da sich die Berechnung aller
möglichen gemeinsamen Verteilungen $\gamma \sim \Pi(P_r, P_\theta)$ etwas schwierig
gestaltet, formt \cite{arjovsky2017wasserstein} die Definition unter
Berücksichtigung der Kontorovich-Rubinstein-Dualität um, sodass
\[
W(P_r, P_\theta) = \sup_{\|f\|_L \leq 1} \mathbb{E}_{x \sim P_r}\left[f(x)\right] - \mathbb{E}_{x \sim P_\theta}\left[f(x)\right]
\]

gilt, wobei das Supremum über alle 1-Lipschitz-Funktionen $f : X \to
\mathbb{R}$ ist. Zusätzlich wird ein kleiner Trick angewendet, um das Problem
weiter zu vereinfachen, indem K-Lipschitz-kontinuierliche Funktionen verwendet
werden.
\[
K \cdot W(P_r, P_\theta) = \sup_{\|f\|_L \leq K} \mathbb{E}_{x \sim P_r}\left[f(x)\right] - \mathbb{E}_{x \sim P_\theta}\left[f(x)\right]
\]

Nehmen wir nun an, dass die Abbildung $f \in \left\{f_w\right\}_{w \in W}$
parametrisiert durch $w$ existiert, wobei $W$ die Menge aller möglichen
Parameter darstellt, so können die Parameter $w$ und damit die Abbildung $f_w$
von einem neuronalen Netz erlernt werden, um so die Wasserstein-Distanz
effizient abzuschätzen. Hier bildet der Wasserstein-Abstand also gleichzeitig
die Loss-Funktion des Kritisierer mit
\[
W(P_r, P_\theta) = \max_{w \in W} \mathbb{E}_{x \sim P_r}\left[f_w(x)\right]
- \mathbb{E}_{z \sim P_r(z)}\left[f_w(g_\theta(z))\right].
\]

Trotzdem darf nicht vergessen werden, dass dies nur gültig ist, falls die
Funktion 1-Lipschitz-kontinuierlich ist. Um dies zu erzwingen, werden die
Werte der aktualisierten Gewichte des Kritisierer zwischen $\left[-c; c\right]$
gehalten. Dabei muss laut \cite{arjovsky2017wasserstein} $c$ relativ klein
sein.

\begin{algorithm}
\SetAlgoLined
\KwIn{Lernrate $\alpha$, Clipping-Parameter $c$, Batch-Größe
$m$, Anzahl von Kritisierer-Iterationen $n_{critic}$.}
\KwResult{Trainieren der Kritisierer-Parameter
$w$ und Generator-Parameter $\theta$.}
\caption{Wasserstein GAN nach \cite{arjovsky2017wasserstein}. Standardwerte
für die Eingabeparameter sind $\alpha = 5\cdot10^{-5}, c = 0.01, m = 64$
und $n_{critic} = 5$.}
\label{alg:wgan}
\BlankLine

\While{$\theta$ \textnormal{ist nicht konvergiert}}{
    \For{$t = 0, ..., n_{critic}$}{
    Erzeuge Batch $\left\{x_i \;\lvert\; 1 \leq i \leq m\right\} \sim
    \mathbb{P}_r$ aus realen Daten\;
    Erzeuge Batch $\left\{z_i \;\lvert\; 1 \leq i \leq m\right\} \sim
    \mathbb{P}_z$ aus latenten Vektoren\;
    \BlankLine
    $g_w \leftarrow
    \nabla_w \left[ \frac{1}{m}\sum_{i=1}^{m} f_w(x_i) -
    \frac{1}{m}\sum_{i=1}^{m} f_w(g_\theta(z_i)\right]$\;
    \BlankLine
    $w \leftarrow w + \alpha \cdot \mathrm{RMSProp}(w,
    g_w)$\;
    $w \leftarrow \mathrm{clip}(w, -c, c)$\;
    }
    \BlankLine
    Erzeuge Batch $\left\{z_i \;\lvert\; 1 \leq i \leq m\right\} \sim
    \mathbb{P}_z$ aus latenten Vektoren\;
    $g_\theta \leftarrow -\nabla_\theta \left[\frac{1}{m} \sum_{i=1}^{m}
    f_w(g_\theta(z_i))\right]$\;
    $\theta \leftarrow \theta - \alpha \cdot \mathrm{RMSProp}(\theta,
    g_\theta)$\;
}
\end{algorithm}

Zusätzlich ist bei Wasserstein-GANs von einem Kritisierer (Critic) anstatt
eines Diskriminators die Rede. Der Grund dafür ist, dass ein Diskriminator
zwischen \textit{fake} und \textit{real} unterscheidet, mehr nicht. Der
Kritisierer führt diese Unterteilung der Eingabeparameter nicht durch, sondern
bewertet bzw. kritisiert diese viel mehr. Mathematisch ausgedrückt sprechen
wir hierbei von einer linearen Ausgabe in $\mathbb{R}$ im Falle des
Kritisierers, während der Diskriminator eine binäre Ausgabe erzeugt.  Zu
Beginn von Algorithmus \ref{alg:wgan} werden die Parameter $w$ für den
Kritisierer und $\theta$ für den Generator initialisiert. Anschließend werden
$m$ Datenpunkte bzw. ein Batch aus dem reellen Datensatz (Verteilung
$\mathbb{P}_r$) gezogen. Dies muss nicht unbedingt zufällig sein. Auch werden
$m$ zufällige Vektoren erzeugt, die als Eingabe für den Generator dienen,
welcher wiederum Fake-Daten erzeugt. Dabei bilden die Ausgaben des Generators
eine eigene Verteilung $\mathbb{P}_\theta$.  Ziel des Generators ist es nun,
die Distanz zwischen den beiden Verteilungen $\mathbb{P}_r, \mathbb{P}_\theta$
zu minimieren, um möglichst realitätsnahe Ausgaben erzeugen zu können. Als
nächstes werden die Gradienten $g_w$ für Parameter $w$ mithilfe von
Gradient-Descent, dargestellt als $\nabla_w$, berechnet. Hierfür wird die
Wasserstein-Distanz als Loss-Funktion verwendet.  Der nächste Schritt besteht
daraus, die Parameter $w$ des Kritisierer-Netzwerks mithilfe des
RMSprop-Algorithmus zu aktualisieren und die aktualisierten Gewichte so gering
wie möglich zu halten, um die K-Lipschitz-Kontinuität zu gewährleisten. Dies
wird mithilfe der Funktion $\mathrm{clip}$ umgesetzt, welche die
Parameterwerte in einem bestimmten Intervall $\left[-c; c\right]$ festsetzt.
Die bis hier erläuterten Schritte werden $n_{critic}$-mal durchgeführt, sodass
das Kritisierer-Netzwerk immer öfter trainiert wird, als der Generator. Dieser
wird nun optimiert, indem wieder $m$ Vektoren zufällig erzeugt und als Eingabe
für das Generator-Netzwerk verwendet werden. Der Generator erzeugt damit $m$
zufällige Ausgaben, die wiederum als Eingaben in das Kritisierer-Netzwerk
gegeben werden. Aus den Ausgaben wird dann der Mittelwert gebildet und zum
Bestimmen der Gradienten von den Generator-Parametern $\theta$ verwendet.

Im direkten Vergleich zu dem originalen GAN \cite{goodfellow2014generative}
werden einige Änderungen in der Architektur vorgenommen. Während in dem
originalen Anstatz fast nach jeder Schicht eine Batch-Normalization
vorgenommen wird, können diese bei WGANs entfallen. Standard-GANs würden
hierbei kaum interpretierbare Resultate erzeugen, WGANs hingegen produzieren
trotzdem gute Ergebnisse, wie Experimente von \cite{arjovsky2017wasserstein}
zeigen (siehe Abbildung \ref{fig:wgan-gan-no-batchnorm}). Das Wasserstein-GAN
hat zusätzlich noch einige nützliche Eigenschaften. So wird unter anderem
durch Annäherung des Wasserstein-Abstandes zwischen Generator- und
Ausgangsdistribution das Problem des Mode-Collapse gelöst.  Durch den
Wasserstein-Abstand wird der Abstand zwischen den Verteilung wesentlich besser
minimiert (im Falle des Generator-Modells) als bei der KL- oder JS-Divergenz.
Die Ausgabe eines Kritisierers stellt außerdem eine Bewertung der Eingabe dar,
anstatt diese in die Kategorien Fälschung oder Realität einzuteilen und besitzt
deshalb mehr Aussagekraft.

\begin{figure}
\includegraphics[width=0.5\textwidth]{images/image-022.png}
\includegraphics[width=0.5\textwidth]{images/image-024.png}
\caption{Vergleich von WGAN und Standard-GAN \cite{arjovsky2017wasserstein}.
    Links sind Ausgaben vom WGAN-Algorithmus zu sehen während rechts Ausgaben
    eines Standard-GANs dargestellt sind. In beiden Generator-Modellen wurden
    Batch-Normalization-Layer entfernt. Klar zu erkennen ist, dass WGAN immer
    noch interpretierbare Ergebnisse liefert während bei Standard-GANs Probleme
    erkennbar sind.}
\label{fig:wgan-gan-no-batchnorm}
\end{figure}

\subsection{Wasserstein-GAN mit Gradient-Penalty}
Ein großes Problem von Wasserstein-GANs ist das Clippen der Gewichte in ein
fest definiertes Intervall, um die 1-Lipschitzstetigkeit zu erfüllen. Das dies
keine elegante Lösung ist, liegt auf der Hand. In \cite{gulrajani2017improved}
wurde speziell dieses Problem genauer untersucht und es wurde festgestellt,
dass das Beschneiden der Gewichte den Kritisierer dazu verleitet, nur extrem
einfache Funktionen zu erlernen, wie der Vergleich in Abbildung
\ref{fig:problems-of-weight-clipping} zeigt. 

Um das Problem des Weight-Clippings anzugehen, stellt
\cite{gulrajani2017improved} eine alternative Lösung vor, die auf anderem Wege
die 1-Lipschitzstetigkeit in WGANs sicherstellen soll. Hierbei soll
Gradient-Penalty helfen und wird als
\[
(\|\nabla_{\hat{x}} D(\hat{x})\| - 1)^2
\]
berechnet, wobei $\hat{x} =  x \epsilon + \tilde{x}(1 - \epsilon)$ eine
zufällige Gewichtung zwischen realen ($x$) und generierten Daten ($\tilde{x}$)
darstellt. Das $\epsilon$ wird dabei zufällig aus $\left[0, 1\right]$ gewählt.
Daraus resultierend gestaltet sich die neue Loss-Funktion des Kritisierers wie
folgt.
\[
L = \mathbb{E}_{\tilde{x} \sim \mathbb{P}_g}\left[D(\tilde{x})\right] -
    \mathbb{E}_{x \sim \mathbb{P}_r}\left[D(x)\right] +
    \lambda \cdot \mathbb{E}_{\hat{x} \sim \mathbb{P}_{\hat{x}}}\left[(\|\nabla_{\hat{x}} D(\hat{x})\| - 1)^2\right]
\]
Der Kritisierer ist durch diese Änderung nun wesentlich besser dazu in der
Lage, komplexere Verteilungen zu erlernen.

\begin{algorithm}
\caption{WGAN mit Gradient-Penalty \cite{gulrajani2017improved}.}
\KwIn{Gradient-Penalty-Koeffizient $\lambda$, Anzahl von Kritisierer-Iterationen
$n_{critic}$, Batch-Größe $m$, Adam-Hyperparameter $\alpha, \beta_1,
\beta_2$.}
\KwResult{Trainieren der Kritisierer-Parameter $w$ und Generator-Parameter
$\theta$.}
\BlankLine
\While{$\theta$ \textnormal{ist nicht konvergiert}}{
    \For{$t = 1, ..., n_{critic}$}{
    \For{$i = 1, ..., m$}{
        Wähle reale Probe $x \sim \mathbb{P}_r$, latenten Vektor $\vec{z} \sim
        \mathbb{P}_z$, zufällige Zahl $\epsilon \in \left[0, 1\right]$\;
        $\tilde{x} \leftarrow G_\theta(\vec{z})$\;
        $\hat{x} \leftarrow x\epsilon + \tilde{x}(1 - \epsilon)$\;
        $L_i = D_w(\tilde{x}) - D_w(x) + \lambda(\|\nabla_{\hat{x}}
        D_w(\hat{x})\| - 1)^2$\;
    }
    \BlankLine
    $w \leftarrow \mathrm{Adam}(\nabla_w \frac{1}{m} \sum_{i=1}^m L_i, w,
    \alpha, \beta_1, \beta_2)$\;
    }
    \BlankLine
    Wähle einen Batch aus latenten Vektoren $\{\vec{z}_i\}_{i=1}^m \sim
    \mathbb{P}_z$\;
    $\theta \leftarrow \mathrm{Adam}(\nabla_{\theta} \frac{1}{m} \sum_{i=1}^m
    - D_w(G_\theta(\vec{z}_i)), \theta, \alpha, \beta_1, \beta_2)$\;
}
\end{algorithm}

\begin{figure}
\centering
\includegraphics[width=\textwidth]{images/problems_of_weight_clipping}
\caption{Vergleich zwischen Weight-Clipping (oben) und Gradient-Penalty
(unten). Man erkennt deutlich, dass die Separierung der Ausgangsverteilung,
dargestellt durch orangene Punkte, durch Weight-Clipping in sehr
vereinfachte Funktionen resultiert. Gradient-Penalty lässt hingegen
komplexere Strukturen von Verteilungen zu \cite{gulrajani2017improved}.}
\label{fig:problems-of-weight-clipping}
\end{figure}

\subsection{Conditional-Wasserstein-GAN mit Gradient-Penalty}\label{section:cwgan-gp}
Generative Modelle wie das WGAN oder DCGAN erzeugen bei einem zufälligen
Eingabevektor immer ein zufälliges, interpoliertes Element aus der
Trainingsdistribution. Besitzt der Trainingsdatensatz verschiedene Klassen, so
werden diese Modelle auch zwischen den Klassen interpolierte Ergebnisse
erzeugen. Aber was ist, wenn wir ein zufälliges Element einer bestimmten Klasse
erzeugen wollen? Nehmen wir als Beispiel einen Datensatz bestehend aus Bildern
von Hunden und Katzen. Die bisher vorgestellten DCGANs und WGANs würden auf
diesen Datensatz trainiert werden und würden bei einem zufälligen Eingabevektor
Hunde oder Katzenbilder bzw. eine Mischung aus beidem erzeugen. Möchte man nun
jedoch den Generator des GANs dazu bringen, nur zufällige Hundebilder zu
erzeugen, so muss aufwändig die dafür zuständige Komponente des Eingabevektors
extrahiert werden, die bestimmt, ob der Generator ein Hunde- oder ein Katzenbild
erzeugen soll. Eine einfachere Möglichkeit besteht darin, dem Generator
zusätzlich zum latenten Vektor die Klasse der zu erzeugenden Ausgabe zu
übergeben. Diese Idee wurde erstmals mit
Conditional-Generative-Adversarial-Networks (cGANs) \cite{mirza2014conditional}
umgesetzt und stellt eine Erweiterung zu den bisher besprochenen GANs dar.

Wie bereits erwähnt, wird bei cGANs zusätzlich zum latenten Vektor $x$ das Label
bzw. die Klasse $y$ als Eingabe verwendet. Wie dies technisch umgesetzt wird,
soll das Schema in Abbildung \ref{fig:cgan} verdeutlichen. Die beiden Eingaben,
latenter Vektor und Label, werden zuerst mithilfe einer Vollverbindungsschicht
(Dense) und Umformungsschicht (Reshape) auf die gleiche Form gebracht, da diese
nicht unbedingt gleich aufgebaut sein müssen. Anschließend werden die nun
zueinander passenden Eingaben mithilfe einer Verkettungsschicht (Concat)
aneinandergehängt. Der restliche Verlauf verhält sich wie bei den anderen
vorgestellten GANs, also das Hochskalieren der kleinen Eingabegröße mithilfe von
mehreren aneinandergereihten transponierten Faltungsschichten (Conv2D
Transpose). Die Verlustfunktion für das Trainieren des cGANs muss aufgrund der
Verkettung beider Eingaben ebenfalls angepasst werden. 
\[
\min_G \max_D V(G, D) = \mathbb{E}_{x \sim p_{data}(x)}\left[ \log D(x \lvert y) \right] + \mathbb{E}_{z \sim p_z(z)}\left[ \log (1 - D(G(z \lvert y))) \right]
\]

Der Operator $\lvert$ stellt dabei die Verkettungsoperation dar, sodass $x
\lvert y$ bedeutet, dass $y$ an $x$ gehängt wird. Aus diesem Grund müssen $x$
und $y$ auch dieselbe Form aufweisen. Andernfalls kann die Verkettung nicht
durchgeführt werden. Das komplette Netzwerk lernt damit eine Ausgabe zu
erzeugen, die auf das eingegebene Label passt. Soweit zur Idee von
Conditional-Generative-Adversarial-Networks. Dies soll nun auf Wasserstein-GANs
übertragen werden, sodass in späteren Implementationen
Conditional-Wasserstein-GANs (cWGAN) zur Verfügung stehen, um automatisiert
Datensätze zu erzeugen. Die angepasste Verlustfunktion für den Kritisierer,
welche die Verkettung des Labels berücksichtigt, bildet sich wie folgt.
\[
L_w = \mathbb{E}_{\tilde{x} \sim \mathbb{P}_g}\left[D(\tilde{x} \lvert y)\right] -
    \mathbb{E}_{x \sim \mathbb{P}_r}\left[D(x \lvert y)\right] +
    \lambda \cdot \mathbb{E}_{\hat{x} \sim \mathbb{P}_{\hat{x}}}\left[(\|\nabla_{\hat{x}} D(\hat{x} \lvert y)\| - 1)^2\right]
\]

Der Verlust des Generators lässt sich mit
\[
L_\theta = -\mathbb{E}_{\tilde{x} \sim \mathbb{P}_g} \left[D(\tilde{x} \lvert y)\right]
\]
berechnen.

Für ein Training, in welchem die Daten in Batches vorliegen, wird einfach der
arithmetische Mittelwert berechnet.  Zur Erinnerung, $x \sim \mathbb{P}_r$
entspricht einer Probe $x$ aus einem realen Datensatz. Das $x$ ist damit
ebenfalls real. Zudem sind $\tilde{x} = G(z \lvert y)$ und $\hat{x} = x \epsilon
+ \tilde{x} (1 - \epsilon)$ generierte bzw.  interpolierte Proben. Mithilfe
dieser Verlustfunktionen kann nun ein Conditional-Wasserstein-GAN mit
Gradient-Penalty implementiert werden. Besonders wird dies im nächsten Kapitel
besprochen, wenn ein Generator zum Erzeugen von ganzen Datensätzen mit
verschiedenen Klassen umgesetzt werden soll. 

\begin{figure}
    \centering
    \includegraphics[width=0.8\textwidth]{images/cgan.pdf}
    \caption{Schematische Darstellung eines einfachen Conditional-GANs für den MNIST-Datensatz. Die Eingabe sind ein latenter Vektor mit 100 Komponenten und ein Labelvektor mit 10 Komponenten (entsprechend der Anzahl der Klassen).}
    \label{fig:cgan}
\end{figure}

  \chapter{Erstellen eines Datensatzes}\label{chapter:dataset}
\section{Rahmenbedingungen}
\section{Verwendung von GANs}
\section{Durchführung von Experimenten mit unterschiedlichen GANs}
\section{Analyse der Ergebnisse aus den Experimenten}
  \chapter{Bewegungserkennung}\label{chapter:motion-detection}
In diesem Kapitel sollen die Ergebnisse aus Kapitel \ref{chapter:dataset}
verwendet werden, um eine Bewegungserkennung zu entwickeln. Genauer gesagt soll
auch hier ein Modell entworfen werden, um diese Bewegungen zu erkennen und
klassifizieren. Zudem soll ein Ausblick auf Techniken gegeben werden, um weitere
bewegungsabhängige Eigenschaften mithilfe von KNNs zu extrahieren. Zum Beispiel
ist beim Ausüben von sportlichen Aktivitäten neben der Bewegungsart auch
interessant, ob die Bewegung richtig ausgeführt wurde. Aber auch das
Vorhersagen von zukünftigen Bewegungen anhand kürzlich getätigten Posen kann für
einige Anwendungen hilfreich sein. So auch zum Beispiel beim vorzeitigen
Erkennen von aggressiven Verhaltensmustern, sodass eingegriffen werden kann,
bevor eine kriminelle Tätigkeit ausgeführt werden kann. 

Eine Bewegung kann durch die Änderung des Ortes über die Zeit beschrieben
werden. Das bedeutet, dass eine Fotoaufnahme einer sich bewegenden Person nicht
ausreicht, um die Bewegung aufzuzeichnen. Entsprechend müssen Bilder über Zeit
aufgenommen werden. Die nächste Frage, die sich ergibt ist, wie viele
Bildaufnahmen nötig sind, um eine Bewegung erkennen zu können. Dies hängt
natürlich von der Art der Bewegung und der Bewegungsgeschwindigkeit ab. Möchte
man beispielsweise einen Jumping-Jack aufnehmen und die Person bewegt sich viel
zu langsam, dann sind wesentlich mehr Aufnahmen nötig, als wenn diese sich in
einem normalen Tempo bewegen würde. In der Informationstheorie wird dies
mithilfe der Abtastrate beschrieben, die angibt, wie oft pro Sekunde abgetastet
werden soll.  Damit in Verbindung kann man die minimale Rate durch das
Niquist-Shannon-Abtasttheorem bestimmen, welches aussagt, dass ein Signal exakt
rekonstruierbar ist, wenn ein Signal mit der doppelten Abtastrate $f_{abtast} =
2 \cdot f_{max}$ abgetastet wird.

Ein durchschnittlicher Fahrradfahrer schafft eine Trittfrequenz von maximal 60
Umdrehungen pro Minute während Leistungssportler bis zu 110 Umdrehungen pro
Minute schaffen \cite{smolik}. Das würde bedeuten, dass die Abtastrate
mindestens 2 Hz betragen muss, um das Signal rekonstruierbar aufzuzeichnen.
Betrachtet man nun eine Beinpressbewegung, die ebenfalls mit 60 Tritten pro
Minute ausgeführt wird, ergibt sich ebenfalls eine Abtastfrequenz von 2 Hz. Hier
wird folgendes Problem ersichtlich. Tastet man beide Signale mit 2 Hz ab, so
kann man nicht zwischen Kreisbewegung und Linearbewegung unterscheiden, wenn
sich die abgetasteten Punkte überschneiden. Würde man hingegen mit 3 Hz
abtasten, so wären die Bewegungen eindeutig voneinander unterscheidbar. Dies hat
unter anderem damit zu tun, dass eine lineare Funktion durch zwei Punkte und ein
Kreis stets durch drei sich nicht auf einer Geraden befindenden Punkte
beschreibbar ist. Damit also ein neuronales Netz eine Bewegungserkennung durchführen
kann, ist es wichtig, dass dieses mit möglichst vielen Frames pro Sekunde (FPS)
ausgeführt wird, um verschiedenste Bewegungen erkennen zu können.

Für die Bewegungserkennung wird in dieser Arbeit die Erkennung von menschlichen
Posen eingesetzt, die Schlüsselpunkte des Körpers aus Bildern extrahiert. Aus
diesen Informationen können bereits einige Eigenschaften einer Pose bzw.
Bewegung abgeleitet oder berechnet werden. So ist zum Beispiel die Berechnung
des Winkels zwischen Oberschenkel und Wade relativ simpel, wenn die
entsprechenden Schlüsselpunkte bekannt sind. Dem Benutzer kann dadurch
berechnet werden, ob eine Übung, die abhängig von diesem Winkel ist, richtig
ausgeführt wird oder nicht. Grundsätzlich wird folgende Idee zum Erkennen bzw.
Analysieren von Bewegungen in Betracht verfolgt.  Die von einer Kamera
aufgenommenen Bilder werden als Eingabe in einen Pose-Detektor gegeben, welcher
zunächst die in dem Eingabebild enthaltenen Schlüsselpunkte ausgibt.  Diese
werden dann anschließend in einen angehängten Prediction-Head gegeben, welcher
dann aus den Schlüsselpunkten die Bewegungsart erkennt.

\section{Erkennung von menschlichen Posen}
Das Erkennen von menschlichen Posen hat in den letzten Jahren viel
Aufmerksamkeit bekommen und wird in 2D- bzw. 3D-Human-Pose-Estimation (HPE)
unterschieden. Diese Arbeit beschäftigt sich mit der 2D-HPE, um Schlüsselpunkte
des menschlichen Körpers in Bildern zu detektieren. Auch wird sich auf die
Erkennung von Einzelpersonen in Bildern konzentriert, anstatt viele
verschiedene Personen gleichzeitig zu bestimmen. Hierbei spricht man von
Single-Person-Pipelines, die wiederum in zwei unterschiedliche Methoden
unterteilt werden: Regressions- und Körperteildetektionsmethoden
\cite{zheng2021deep}.

Mit Regressionsmethoden werden zu Eingabebildern direkt die Schlüsselpunkte
ausgegeben. Dabei soll ein KNN lernen, Bilder auf Schlüsselpunktkoordinaten
abzubilden. Die Ausgabe besitzt dementsprechend $k \cdot n$ Komponenten, wobei
$k$ die Anzahl der Schlüsselpunkte und $n$ die Anzahl der Eigenschaften pro
Schlüsselpunkt angibt. Das Problem bei dieser Methode ist, dass Körperteile wie
Hände, Augen und Füße in einem Bild sehr klein ausfallen können. Dort ist es
dann sehr schwer, diese von der Umgebung unterscheiden zu können und damit eine
präzise Aussage darüber zu treffen, wo sich deren Schlüsselpunkte befinden.
Auch ist es möglich, dass einige Körperteile nicht auf dem Bild zu sehen sind.
Das Treffen einer Aussage, ob das entsprechende Körperteil überhaupt im Bild
vorhanden ist, gestaltet sich bei dieser Methode eher schwierig. Hierbei kann
die folgende Methode helfen.

Bei Körperteildetektionsmethoden verläuft die Erkennung von Schlüsselpunkten ein
wenig anders. Anstatt direkt auf Koordinaten der Schlüsselpunkte abzubilden,
versucht man hier eine Reihe von Wärmekarten zu erzeugen. Genauer ausgedrückt
wird für $k$ Schlüsselpunkte eine Menge aus Wärmekarten $H = \{H_1, ..., H_k\}$
erzeugt. Eine Wärmekarte $H_i$ enthält dabei für jedes Koordinatenpaar $(x, y)$
einen Wahrscheinlichkeitswert $p_{y, x} \in H_i$, welcher angibt, zu welcher
Wahrscheinlichkeit dort ein Schlüsselpunkt vorliegt.  Für das Training solcher
Modelle wird meistens die Abweichung zwischen generierten und tatsächlichen
Wärmekarten mit Hilfe des Mean-Squared-Errors (MSE) minimiert. Grund\-sätz\-lich
besitzen Körperteildetektionsmethoden den Vorteil, dass auch relativ kleine
Körperteile gut erkannt werden. Der Unterschied zwischen den Methoden ist in
Abbildung \ref{fig:pose-detection} zu sehen. Es werden in aktuellen Forschungen
bezüglich der Posenerkennung häufig Modelle, die Wärmekarten erzeugen, verwendet
\cite{zheng2021deep}.

\begin{figure}
    \includegraphics[width=\textwidth]{images/pose_detection.pdf}
    \caption{Regressions- und Körperteildetektions-Verfahren
    zum Be\-stim\-men von Schlüs\-sel\-punk\-ten eines Menschen. A) Ein Regressor
    versucht direkt die Punkte des menschlichen Körpers als Koordinaten
    auszugeben. B) Ein Körper\-teil\-detektor erzeugt eine Wärmekarte für jeden
    Schlüsselpunkt. Die Werte in den Karten geben die Wahrscheinlichkeit für den
    Aufenthalt eines Punktes an. Die Schlüsselpunkte müssen nach der Detektion aus den Karten dekodiert werden.}
    \label{fig:pose-detection}
\end{figure}

Ein Forschungsteam von Google hat ein neues Machine-Learning-Modell namens
\textit{MoveNet} \cite{movenet} präsentiert, das unter anderem auf mobilen
Geräten ausgeführt werden kann und dabei eine Echtzeiterkennung von menschlichen
Posen ermöglicht. Die Architektur besteht aus einem MobileNetV2-Backbone mit
angehängtem Feature-Pyramid-Network (FPN) und insgesamt vier Prediction-Heads,
die CenterNets \cite{zhou2019objects} sind. Trainiert wurde das Netzwerk
auf den COCO-Datensatz und erreicht laut den Autoren 30 oder mehr FPS auf
modernen Computern, Laptops und Smartphones. Auch dieses Modell verwendet
Wärmekarten, um mögliche Positionen von menschlichen Schlüsselpunkten zu
bestimmen und wird von einem Prediction-Head übernommen. Ein weiterer ist dafür
zuständig, den Abstand (Offset) jedes Pixels als Vektor von der geschätzten
Schlüsselpunktposition zu bestimmen. Ein anderer Kopf übernimmt die Gruppierung
der Schlüsselpunkte und ordnet diesen Personeninstanzen zu. Der vierte Kopf
schätzt das Zentrum der Instanzen. Die vollständige Architektur von MoveNet ist
in Abbildung \ref{fig:movenet-architecture} zu sehen. MoveNet verwendet also
neben dem Erzeugen von Körperteildetektionsmethoden zum Erzeugen von Wärmekarten
auch Regressionsmethoden, um die Position von Schlüsselpunkten noch genauer
angeben zu können.

\begin{figure}
    \includegraphics[width=\textwidth]{images/movenet_architecture.png}
    \caption{Architektur von MoveNet \cite{movenet}. Der Feature-Extractor
    besteht aus einem vortrainierten MobileNetV2-Backbone mit angehängtem FPN.
    Insgesamt werden vier Prediction-Heads zum Bestimmen von Schlüsselpunkten
    verwendet.}
    \label{fig:movenet-architecture}
\end{figure}

\section{Erkennung von Bewegungsarten}
Bei der Erkennung von Bewegungsarten handelt es sich um ein
Klassifizierungsproblem, bei der es darum handelt, eine Bewegungsaufzeichnung
einer Klasse zuzuordnen. Zu einer beliebigen Bewegungssequenz soll also eine
Aussage getroffen werden, ob es zum Beispiel Hantelübungen sind. Im Laufe dieses
Abschnitts soll dementsprechend ein Machine-Learning-Modell entworfen werden,
welches auf diese Aufgabe trainiert wird. Hierzu muss vorerst definiert werden,
welche Daten verwendet werden und wie diese für das Training vorbereitet werden
können. Damit eine Ortsänderung eines zu betrachtenden Objektes aufgezeichnet
werden kann, müssen mehrere Bilder bzw. Videos aufgezeichnet werden. Eine
Bewegung anhand eines einzelnen Bildes ist nicht möglich, da die Information
über die Ortsänderung dann nicht vorhanden ist. Es gilt also einen Datensatz zu
finden, welcher verschiedene Bewegungen im Videoformat bereitstellt. Alternativ
muss ein entsprechender Datensatz ausgehoben werden, was einen relativ hohen
Zeitaufwand bedeuten würde. Es wird aufgrund seiner Einfachheit der
UCF101-Datensatz \cite{ucf101} verwendet. Dieser besteht aus 101 Bewegungsarten
in Form von Videos, die öffentlich auf der Plattform YouTube zur Verfügung
stehen. Ein Videoclip ist dabei eindeutig einer Bewegungsklasse zugeordnet. Um
den Datensatz künstlich zu erweitern bzw. um eine Augmentation durchzuführen,
wird das KpGAN aus Kapitel \ref{chapter:gans} verwendet. Dieses erzeugt während des Trainings Schlüsselpunktdaten auf Nachfrage.

Es wird nun ein Modell entworfen, welches $n$ Frames eines Videos einliest und
die geschätzte Bewegungsklasse für diese Frames ausgibt. Dabei soll das Netzwerk
lernen, menschliche Posen aus den Frames zu extrahieren und anschließend zu
klassifizieren. Da das Neutrainieren der Posenerkennung zu lange dauern würde,
wird ein vortrainiertes MoveNet als Backbone verwendet. Für die Klassifizierung
werden verschiedene Köpfe implementiert, die anschließend miteinander verglichen
werden. Ziel ist es herauszufinden, welche Architekturen für den Gebrauch auf
mobilen Plattformen geeignet sind. Als Testplattform wird, wie einleitend beschrieben, Android verwendet.

Die zu betrachtenen Metriken sind Genauigkeit und
Aus\-führ\-ungs\-ge\-schwin\-dig\-keit. Als erstes werden einfache
Fully-Connected-Layer verwendet und evaluiert. Es wurden dabei Schritt für
Schritt Änderungen an der Anzahl der Neuronen und Schichten vorgenommen.
Anschließend wird untersucht, ob das Verwenden CNNs ähnliche oder bessere
Ergebnisse liefert, besonders in Hinblick auf die Performance. In Tabelle
\ref{table:motion-detection} sind die durchgeführten Experimente aufgelistet.

Das Ergebnis aus den Trainingsversuchen mit rein-realen Datensätzen ist, dass
CNNs die Bewegungen wesentlich besser erkennen können als die
Fully-Connected-Modelle. Nicht nur, dass sie hier eine höhere Genauigkeit
aufweisen, sie lernen die Verteilung auch wesentlich schneller. Während die
Fully-Connected-Modelle 300 Epochen für eine Genauigkeit von 90-97\% benötigen,
benötigen die CNNs lediglich 9-80 Epochen für bessere Resultate. Der einzige
Nachteil besteht in ihrer Ausführungsgeschwindigkeit, die bis zu vier mal
kleiner ist.

Erstaunlicherweise verhält sich dies bei den Trainingsversuchen mithilfe eines
GANs ein wenig anders. Grundsätzlich erreichten die Fully-Connected-Modelle eine
wesentlich höhere Genauigkeit nach sehr viel weniger Trainingsepochen. Die
Genauigkeit wurde dabei nicht mithilfe von synthetischen Daten des GANs
gemessen, sondern es wurden hierfür wieder echte Daten verwendet. Damit wurde
sichergestellt, dass die Genauigkeiten der verschiedenen Trainingsmethoden
vergleichbar gehalten werden. Das Verwenden von künstlich erzeugten Daten
liefert in diesem Fall also immer bessere Ergebnisse. Dies liegt vermutlich
unter anderem daran, dass GANs die Eigenschaft besitzen, zwischen echten Daten
zu interpolieren und damit automatisch eine Augmentation durchzuführen. Dies
führt dazu, dass die Modelle eine höhere Vielfalt lernen können. Diese Vielfalt
ist im echten Datensatz nicht so ausgeprägt. Hierbei ist wichtig zu erwähnen,
dass das Daten erzeugende GAN auf eben diesen Datensatz trainiert wurde. Der
Datensatz wurde also verbessert, indem ein GAN verwendet wurde, die Verteilung
zu lernen und zu interpolieren, um eine größere Datenvielfalt zu erzeugen.

\begin{table}
    \footnotesize
    \begin{tabularx}{\textwidth}{l|X|c|c|c|c|c}
        \hline
        name & layers & epochs & $p_\mathrm{real}$ & epochs & $p_\mathrm{gan}$ & inference speed \\ \hline

        \multirow{3}{*}{linear} & Dense, 128 \newline ReLU \newline Dense, 3 & 300 & 0.90 & 82 & 0.99 & \num{0.59e-6}s \\ \cline{2-7}

        & Dense, 256 \newline ReLU \newline Dense, 3 & 300 & 0.94 & 49 & 0.99 & \num{0.61e-6}s \\ \cline{2-7}

        & Dense, 128 \newline ReLU \newline Dense, 256 \newline ReLU \newline Dense, 3 & 300 & 0.97 & 70 & 0.99 & \num{0.83e-6}s \\ \hline

        \multirow{3}{*}{conv} & Conv2D, 16, $3 \plh 3$, stride 2 \newline Conv2D, 32, $3 \plh 3$, stride 2 \newline Dense, 3 & 81 & 0.99 & 123 & 0.99 & \num{1.25e-6}s \\ \cline{2-7}

        & Conv2D, 16, $3 \plh 3$, stride 2 \newline Conv2D, 32, $3 \plh 3$, stride 2 \newline Conv2D, 64, $3 \plh 3$, stride 2 \newline Dense, 3 & 44 & 0.99 & 121 & 0.99 & \num{1.74e-6}s \\ \cline{2-7}

        & Conv2D, 16, $3 \plh 3$, stride 2 \newline Conv2D, 32, $3 \plh 3$, stride 2 \newline Conv2D, 64, $3 \plh 3$, stride 2 \newline Conv2D, 128, $3 \plh 3$, stride 2 \newline Dense, 3 & 9 & 0.99 & 126 & 0.99 & \num{2.27e-6}s \\ \hline
    \end{tabularx}
    \caption{Durgeführte Experimente mit verschiedenen Architekturen für die
    Bewegungserkennung. Die Eingabe besteht aus 60 Frames einer
    Bewegungsanimation aus 17 Schlüsselpunkten mit jeweils 3 Eigenschaften (x-,
    y-Koordinaten und Wahrscheinlichkeitsscore) kodiert als Bild (siehe
    Abbildung \ref{fig:motion-images}). Als Ausgabe sollen die Netze die
    geschätzte Klasse, also das Label liefern. $p_\mathrm{real}$ gibt die
    Genauigkeit des jeweiligen Modells an, das mit einem realen Datensatz
    trainiert wurde. $p_\mathrm{gan}$ gibt hingegen die Genauigkeit der Modelle
    an, die mithilfe eines KpGAN-Generators trainiert wurden. Für das Messen der
    Genauigkeit wurden immer reale Daten verwendet.}
    \label{table:motion-detection}
\end{table}

  \chapter{Entwicklung einer mobilen Anwendung}
Auf Basis der ausgearbeiteten Ergebnisse aus Kapitel
\ref{chapter:motion-detection} soll nun eine mobile Anwendung entwickelt werden.
Ziel ist es die theoretischen Schlussfolgerungen mit einem praktischen
Experiment zu verifizieren. Da sich diese Arbeit vor allem mit dem Problem
beschäftigen soll, wie solche Modelle auf mobilen Plattformen überführt werden
können, soll eine Android-App entwickelt werden, um die Ergebnisse
zusammenzufassend zu präsentieren. Android wird als Plattform gewählt, da es zur
Zeit die am häufigsten vertretene mobile Plattform ist und ein entsprechendes
Gerät leicht zur Verfügung steht. Zum Vergleich, Android besitzt einen
Marktanteil von 72,84\%, iOS einen von 26,34\% und 0,82\% werden von sonstigen
Plattformen
gehalten\footnote{https://www.statista.com/statistics/272698/global-market-share-held-by-mobile-operating-systems-since-2009/
(besucht am 13.08.2021)}.

Die Anforderungen an die App sind recht simpel. Es sollen über die Kamera
Bewegungen identifiziert werden, wobei die erkannte Bewegungsart angezeigt
werden soll. Das Zeitfenster, in welches Bewegungen erkannt werden sollen, wird
auf 60 Frames festgelegt. Bei einer Kamera, die 30 Bilder pro Sekunde aufnehmen
kann, wird also vorausgesetzt, dass die Bewegung innerhalb von zwei Sekunden
eindeutig identifizierbar ist. Zusätzlich muss berücksichtigt werden, dass die
Abtastrate für die Bewegungserkennung entsprechend hoch, also die
Ausführungsdauer der Machine-Learning-Modelle möglichst gering sein soll. Der
Grund dafür ist, dass die Modelle den Flaschenhals der Anwendung darstellen und
eine hohe Abtastrate nur mit entsprechend schnellen Netzwerken möglich ist.

\section{Implementierungsdetails}
Der Einfachheit halber werden zwei KNNs verwendet, um eine Bewegung zu
identifizieren. Das eine Netzwerk hat die Aufgabe, Schlüsselpunkte des
menschlichen Körpers in Bildern zu erkennen. Diese werden anschließend in einen
Puffer mit maximal 60 Elementen zwischengespeichert. Wird ein Element in den
Puffer hinzugefügt, so werden alle anderen Elemente zuerst um eine Position nach
hinten (rechts) verschoben. Der erste Slot ist nun dementsprechend leer und wird
von dem neuen Element belegt. Ist der Puffer bereits voll, so wird das letzte
Element entfernt. Ein Element dieses Puffers sind 17 Schlüsselpunkte des
menschlichen Körpers. Dieser Puffer bildet damit die Eingabe für das zweite
Netzwerk.  Dieses hat die Aufgabe, Schlüsselpunkte aus 60 Kamerabildern einer
Bewegungsklasse zuzuordnen.

Für die Schlüsselpunkterkennung wird MoveNets Lightning-Architektur
\cite{movenet} verwendet. Diese wurde speziell für mobile Geräte entwickelt,
sodass die Erkennung von Schlüs\-sel\-punk\-ten in Echtzeit durchgeführt werden
kann. Hier gilt es zu testen, ob die Performance in der Tat ausreichend ist, um
auf modernen, aber leistungsärmeren Geräten in Echtzeit zu laufen. Für die
Bewegungserkennung werden die verschiedenen Architekturen aus Kapitel
\ref{chapter:motion-detection} verwendet und getestet. Auch hier ist zu
überprüfen, ob diese in Zusammenarbeit mit MoveNet zu einer ausreichenden
Performance kommen. Eine ausreichende Performance wird dann angenommen, wenn
entsprechende Bewegungen über die Kamera bzw. über die entwickelte App richtig
erkannt wurden.

Für die Entwicklung der Android-App wurde \textit{Kotlin} verwendet. Zudem wird
die \textit{Ten\-sor\-Flow-Lite-API} (TFLite) verwendet, um die exportierten
Modelle auf Android auszuführen. Die in dieser Arbeit implementierten Modelle
wurden alle mit der \textit{TensorFlow-API} trainiert.  Da es sich dabei jedoch
um ein Desktop Framework handelt, wurden die trainierten Modelle im
\textit{Saved-Model}-Format gespeichert und anschließend in das TFLite-Format
umgewandelt. 

Es soll nun der allgemeine Aufbau der mobilen Anwendung besprochen werden. Im
ersten Schritt wurde sich Gedanken über mögliche Klassen der Anwendung gemacht.
Diese sind im UML-Klassendiagramm in Abbildung \ref{fig:uml-app} zu sehen. Als
Einstiegspunkt in dieser, sowie in jeder anderen Android-App, dient die
\texttt{MainActivity}-Klasse. Hier werden vor allem die View-Elemente der App
initialisiert. Auch die Modelle für das Erkennen von Bewegungen werden hier
geladen. Dies ist zum einen das MoveNet-Lightning-Modell\footnote{MoveNet-Lightning, https://tfhub.dev/google/movenet/singlepose/lightning/4} für die
Schlüsselpunkterkennung und zum anderen das MotionNet aus Kapitel
\ref{chapter:motion-detection}. Warum die beiden Modelle nicht in ein einziges
Modell zusammengefügt worden sind, hat den Grund, dass die Schlüsselpunkte in
den dazugehörigen Puffer gespeichert werden sollen. Dieser wird dann
anschließend vom MotionNet ausgelesen und interpretiert, sodass eine
Bewegungserkennung stattfinden kann. Die Ausgabe von einem
Schlüsselpunktdetektor kann demnach nicht direkt als Eingabe für das MotionNet
dienen. Dies liegt natürlich auch daran, dass versucht wird, eine Bewegung in
Echtzeit zu erkennen. Deshalb ist es notwendig, dass die Bilder der Kamera auch
in Echtzeit analysiert werden. Erst wenn diese Abtastung ein bestimmtes
Zeitfenster beinhaltet, kann eine zeitabhängige Änderung einer Pose erkannt
werden. Die beiden geladenen Modelle werden nun in eine dafür vorgesehene Klasse
verwaltet. \texttt{MotionDetector} soll die Erkennungslogik implementieren, was
unter anderem die Implementierung des Schlüsselpunktpuffers (siehe Abbildung
\ref{fig:camera-frame-buffer}) bedeutet. Diese Klasse wird abstrakt umgesetzt,
damit neue Modelle zum Erkennen von Bewegungen einfach von ihr erben und
entsprechende Methoden überschreiben können, um den Weg in die Anwendung
leichter finden zu können. Als Beispiel hierfür wurde das \texttt{MotionNet}
implementiert, welches nun von der abstrakten Klasse erbt. Hier werden die
Modelle als \textit{Interpreter} der \textit{TensorFlow-Lite-API} verwaltet. Die
Inferenz der Modelle wird schließlich in den zu überschreibenden Methoden
ausgeführt. Damit die Ausführung der Modelle überhaupt stattfinden kann, müssen
Bilder über die Kamera in die Anwendung geleitet werden. Dies wird mithilfe der
implementierten \texttt{CameraManager}-Klasse umgesetzt. Diese stellt das
aktuelle Bild der Kamera über ein dafür vorgesehenes Steuerelement dar und
bietet eine Schnittstelle zum Abfangen einzelner Frames. Diese Frames werden
anschließend über die \texttt{MainActivity} in die Bewegungserkennung geleitet.
Nach der Inferenz wird das User-Interface mithilfe der \texttt{MainActivity}
aktualisiert und die Ergebnisse dargestellt. Speziell für die Darstellung der
Schlüsselpunkte und der Bewegungsart wird die View-Klasse \texttt{PersonOverlay}
umgesetzt. Diese stellt alle aus der Bewegungserkennung resultierenden
Ergebnisse grafisch dar.

\begin{figure}
    \includegraphics[width=\textwidth]{images/camera_frame_buffer.pdf}
    \caption{Schematische Darstellung des Puffers der Bewegungserkennung,
    welcher $n = 60$ Schlüsselpunkte des menschlichen Körpers speichert.}
    \label{fig:camera-frame-buffer}
\end{figure}

\begin{figure}
    \includegraphics[width=\textwidth]{images/app_uml.pdf}
    \caption{UML-Klassendiagramm der mobilen Anwendung zum Testen der
    Machine-Learning-Modelle für die Bewegungserkennung.}
    \label{fig:uml-app}
\end{figure}

\section{Testaufbau und Ergebnisse}
Die Bewegungserkennung aus den vorherigen Abschnitten soll nun mithilfe der
implementierten mobilen Applikation in der Praxis getestet werden. Hierfür muss
vorerst der Testaufbau besprochen werden, sodass die Ergebnisse nachvollziehbar
und reproduzierbar sind. Als Plattform wird Android zusammen mit ausgewählten
mobilen Geräten verwendet. Ziel ist es herauszufinden, ob die Bewegungserkennung
auf modernen, aber auch auf älterer Hardware in Echtzeit ausführbar ist.
  \include{sections/discussion.tex}
  \chapter{Fazit und Ausblick}
In dieser Thesis wurden verschiedene Modelle des maschinellen Lernens entwickelt
und evaluiert. Vor allem wurden sogenannte Generative-Adversarial-Networks
(GANs) erforscht und neue Methoden zum Erzeugen von künstlichen Bewegungen
gesucht, um einen kompletten Datensatz mithilfe eines solchen Modells zu
erzeugen. Als Ergebnis wurde zuerst ViGAN präsentiert. Ziel von ViGAN ist es,
aus einem bestehenden Video-Datensatz neue bisher unbekannte Proben zu erzeugen.
Dies gelang auch bis zu einem gewissen Grad. Es wurden zwar neue Videos vom
Netzwerk erzeugt, jedoch konnte nicht sichergestellt werden, ob sich eine Person
tatsächlich bewegt oder still steht.  So sind Personen oft regungslose starre
Körper, während sich z.B. ganze Räume über die Zeit verändern. Als Alternative
dazu wurde KpGAN entworfen, welches direkt verschiedenste Bewegungsanimationen
erzeugen kann. Diese Bewegungsanimationen bestehen dabei aus menschlichen
Schlüsselpunkten und werden als Bilder kodiert.  Experimente haben gezeigt, dass
Datensätze, die komplett mithilfe von KpGAN erzeugt wurden, mindestens genau so
gut für das Training anderer Modelle geeignet sind, wie entsprechende reale
Datensätze.  Dies bringt einen großen Vorteil mit sich, denn der eigentliche
Datensatz kann ohne großen Aufwand beliebig erweitert werden, während beim
realen Datensatz ein signifikanter Zeitaufwand entstehen würde.

Im Anschluss wurde mithilfe von KpGAN weitere neuronale Netzwerke trainiert, die
Bewegungen aus Schlüsselpunktsequenzen erkennen sollen.  Die Herausforderung ist
dabei die Echtzeiterkennung auf mobilen Geräten. Die verwendeten Metriken waren
Genauigkeit und Performanz. Aus denExperimenten resultierte schließlich das
MotionNet, welches eine sehr gute Erkennrate und Ausführungsgeschwindigkeit
besitzt.  Mithilfe des trainiertenMotionNets und eines externen Modells von
Google (MoveNet-Lightning) wurde anschließend eine mobile App entwickelt, welche
die Anwendbarkeit und Performanz auf mobilen Geräten messen soll. Auf modernen
Smartphones wurde eine Aus\-führ\-ungs\-ge\-schwindig\-keit von bis zu 14 FPS
gemessen.  Damit wurde eindeutig bewiesen, dass eine Bewegungserkennung auf
mobilen Plattformen in Echtzeitausführbar ist.

Weiterführende Arbeiten können sich in Zukunft mit dem Verbessern des ViGAN
be\-schäf\-tigen, um das Problem mit den sich nicht bewegenden Personen in
Videos zu lösen. Zudem kann das MotionNet um weitere Bewegungen erweitert
werden.

  \printbibliography
\end{document}

\documentclass{hsflensburg}
\title{Bewegungserkennung auf mobilen Geräten mit Verwendung von GANs für eine automatische Datensatzgenerierung}
\subtitle{Master-Thesis}

\author{
  \name{Florian Hansen}\\
  \institution{Hochschule Flensburg}
}

\usepackage[ngerman]{babel}

\begin{document}
  \maketitle
  \tableofcontents

  \section{Einleitung}

  \section{Generative Adversarial Networks}
  In Machine-Learning existieren viele verschiedene Modelle, die vorhandene
  Datensätze analysieren und anhand der Daten lernen, Entscheidungen zu treffen.
  Besitzt man beispielsweise einen Datensatz bestehend aus Fotoaufnahmen von
  Tieren, so kann ein Klassifierer trainiert werden, um einem Bild eine
  Tierklasse zuzuweisen. Aus diesem Grund fässt man diese Modelle unter dem
  Begriff \textit{Bildklassifizierung} zusammen.

  Wesentlich interessanter ist das Erkennen von vielen Objekten innerhalb eines Bildes,
  anstatt das gesamte Bild nur einer Klasse zuzuweisen. In der \textit{Objekterkennung}
  entwickelt man Modelle, welche mehr als nur eine Klasse erkennen können. Sie liefern
  zusätzlich zu den erkannten Klassen ihre Position und Größe innerhalb des
  Bildes. 

  Neben Modellen, die zu einem bestimmten Sachverhalt eine Aussage treffen
  können, existieren auch Modelle, welche in der Lage sind, neue Sachverhalte zu
  erzeugen. Diese fallen unter dem Begriff \textit{Generative Adversarial
  Networks} (GANs) und bilden das Hauptthema dieses Abschnitts. Das interessante an
  diesen generativen Modellen ist, dass sie nicht nur die Strukturen eines
  Datensatzes lernen, sondern darüber hinaus neue Elemente der
  Ausgangsdistribution erzeugen können. Trainiert man also ein generatives
  Modell auf einen Datensatz, welcher Bilder von verschiedenen Tieren enthält,
  können neue Bilder der gleichen Art erzeugt werden.

  Aber nicht nur zum Erzeugen von Bildern kann diese Art von Modellen verwendet
  werden. Auch bei Aufgaben, bei denen eine Voraussagung getroffen werden soll,
  werden generative Modelle eingesetzt. 

  \subsection{Mode-Collapse}
  \subsection{Deep Convolution GAN}
  \subsection{Wasserstein GAN}
  \subsection{Wasserstein GAN mit Gradient Penality}
  \subsection{Unrolled GAN}
  \subsection{Least Squares GAN}

  \section{Erstellen eines Datensatzes}
  \subsection{Rahmenbedingungen}
  \subsection{Verwendung von GANs}
  \subsection{Durchführung von Experimenten mit unterschiedlichen GANs}
  \subsection{Analyse der Ergebnisse aus den Experimenten}

  \section{Bewegungserkennung}
  \subsection{Ground-Truth}
  \subsection{Background-Substraction}
  \subsection{Erkennung von Geschwindigkeiten}
  \subsection{Erkennung von Anomalien}
  \subsection{Erkennung von Bewegungsarten}
  \subsection{Vorhersage von Bewegungen}
  \subsection{Architektur einer mobilen Anwendung}

  \section{Fazit und Ausblick}
\end{document}

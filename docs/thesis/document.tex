\documentclass{hsflensburg}
\title{Bewegungserkennung auf mobilen Geräten mit Verwendung von GANs für eine automatische Datensatzgenerierung}
\subtitle{Master-Thesis}

\author{
  \name{Florian Hansen}\\
  \institution{Hochschule Flensburg}
}

\usepackage[ngerman]{babel}

\begin{document}
  \maketitle
  \tableofcontents

  \section{Einleitung}

  \section{Generative Adversarial Networks}
  \subsection{Mode-Collapse}
  \subsection{Deep Convolution GAN}
  \subsection{Wasserstein GAN}
  \subsection{Wasserstein GAN mit Gradient Penality}
  \subsection{Unrolled GAN}
  \subsection{Least Squares GAN}

  \section{Erstellen eines Datensatzes}
  \subsection{Rahmenbedingungen}
  \subsection{Verwendung von GANs}
  \subsection{Durchführung von Experimenten mit unterschiedlichen GANs}
  \subsection{Analyse der Ergebnisse aus den Experimenten}

  \section{Bewegungserkennung}
  \subsection{Ground-Truth}
  \subsection{Background-Substraction}
  \subsection{Erkennung von Geschwindigkeiten}
  \subsection{Erkennung von Anomalien}
  \subsection{Erkennung von Bewegungsarten}
  \subsection{Vorhersage von Bewegungen}
  \subsection{Architektur einer mobilen Anwendung}

  \section{Fazit und Ausblick}
\end{document}

\chapter{Erstellen eines Datensatzes}\label{chapter:dataset}
Ein wichtiger Bestandteil beim Entwickeln von künstlichen neuronalen Netzen ist
der unterliegende Datensatz, der zum Trainieren der Parameter der Netze
verwendet wird. Eine Machine-Learning-Modell ist nur so gut wie der verwendete
Datensatz. Dieser muss deshalb eine große Varianz der repräsentierten Daten
besitzen. Das bedeutet, dass ausreichend Einträge vorhanden sein müssen, um das
Netzwerk auf ähnliche, aber unbekannte Probleme vorzubereiten. Ein beliebter
Datensatz ist beispielsweise der MNIST-Datensatz \cite{6296535}, welcher unter
anderem Bilder von handgeschriebenen Ziffern bereitstellt. Anhand des
MNIST-Beispiels bedeutet eine große Varianz, dass eine Ziffer durch viele
verschiedene Bilder repräsentiert wird, die alle eine andere Perspektive des
gleichen Kontexts darstellen.

In diesem Abschnitt soll nun ein Datensatz eingeführt werden, welcher die Bewegungserkennung unterstützt. Um diesen Datensatz zu erstellen, muss vorab geklärt werden, was eine Bewegung eigentlich ist. Betrachtet man einen Punkt im Raum, dann beschreibt eine Bewegung die Änderung des Ortes eines Objekts über die Zeit.

\section{Rahmenbedingungen}
\section{Verwendung von GANs}
\section{Durchführung von Experimenten mit unterschiedlichen GANs}
\section{Analyse der Ergebnisse aus den Experimenten}
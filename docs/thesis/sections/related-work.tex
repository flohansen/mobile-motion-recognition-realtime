\chapter{Verwandte Arbeiten}
In \cite{yu2018remotenet} wird ReMotENet vorgestellt. Dies ist eine Architektur,
die zum Erkennen und Separieren von Bewegungen eingesetzt werden kann. Hierbei
werden ganze Videoclips eingelesen, wodurch eine erhöhte Geschwindigkeit im
Vergleich zu anderen Methoden wie Background-Substraction und einer
anschließenden Objektdetektion erreicht wird. Leider wird hierbei nur
identifiziert, welches Objekt sich bewegt, aber nicht, welche Bewegung
ausgeführt wird. Auch handelt es sich hierbei nicht um eine Echtzeiterkennung.

Die Arbeit von \cite{kidzinksi2020} präsentiert eine Methode zum Abschätzen von
Bewegungseigenschaften. So werden beispielsweise Laufgeschwindigkeiten,
Kniebeugewinkel und Kadenz einer Bewegung mithilfe eines Deep-Learning-Modells
erfasst. Auch hier handelt es sich nicht um eine Echtzeiterkennung.

In \cite{zhu2020} wird die Erkennung von bewegten Objekten behandelt. Hier
werden vor allem sich bewegende Objekte in Echtzeit und in hochauflösenden
Videoframes detektiert. Das Modell erreicht eine Geschwindigkeit von 21 FPS mit
einer Genauigkeit von 86,15\% bei einer Bildauflösung von $1920 \times 1080$
Pixel. Eine tiefere Bewegungsanalyse wird hier jedoch nicht vorgenommen.

In \cite{erol2020motion} wird auf das Klassifizieren von Bewegungen eingegangen.
Zudem werden synthetische Radar-Mikro-Doppler-Signaturen von Bewegungen von
einem GAN erzeugt, um später mit diesem einen Datensatz für einen Klassifizierer
zu generieren. Insgesamt wird eine Erkennungsrate von 93\% erreicht. Die
Erkennung geschieht dabei nicht in Echtzeit.